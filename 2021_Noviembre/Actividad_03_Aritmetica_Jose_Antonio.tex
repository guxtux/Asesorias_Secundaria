\input{preambulo_materiales}

\title{Ejercicios de repaso - José Antonio \vspace{-2cm}}
\author{}
\date{ }

%--------------------------------------------------------------------
\newcounter{choice}
\renewcommand\thechoice{\Alph{choice})}
%\newcommand\choicelabel{\thechoice.}
\newcommand\choicelabel{\thechoice}

\newenvironment{choices}%
  {\list{\choicelabel}%
     {\usecounter{choice}\def\makelabel##1{\hss\llap{##1}}%
       \settowidth{\leftmargin}{W.\hskip\labelsep\hskip 2.5em}%
       \def\choice{%
         \item
       } % choice
       \labelwidth\leftmargin\advance\labelwidth-\labelsep
       \topsep=0pt
       \partopsep=0pt
     }%
  }%
  {\endlist}

\newenvironment{oneparchoices}%
  {%
    \setcounter{choice}{0}%
    \def\choice{%
      \refstepcounter{choice}%
      \ifnum\value{choice}>1\relax
        \penalty -50\hskip 1em plus 1em\relax
      \fi
      \choicelabel
      \nobreak\enskip
    }% choice
    % If we're continuing the paragraph containing the question,
    % then leave a bit of space before the first choice:
    \ifvmode\else\enskip\fi
    \ignorespaces
  }%
  {}
%----------------------------------------------------------

\begin{document}
\maketitle
\fontsize{14}{14}\selectfont

Resuelve los siguientes enunciados:
\begin{enumerate}[label=\alph*)]
\item Si $14$ libros cuestan $\$ 84$, ¿cuánto costarían $9$ libros? \textbf{R.} \rule{2cm}{0.1mm}
\item Si $25$ trajes cuestan $\$ 250$, ¿cuánto costarían $63$ trajes? \textbf{R.} \rule{2cm}{0.1mm}
\item Si $19$ sombreros cuestan $\$ 57$, ¿cuántos sombreros podría comprar con $\$ 108$? \textbf{R.} \rule{2cm}{0.1mm}
\item Se reparten $84$ kilogramos de víveres entre $3$ familias compuestas de $7$ personas cada una. ¿Cuántos kilogramos recibe cada persona? 
\\
\textbf{R.} \rule{2cm}{0.1mm}
\item Un estanque cuya capacidad es de $300$ litros está vacío y cerrada su descarga. Si abrimos al mismo tiempo tres llaves que vierten: la primera, $36$ litros en $3$ minutos, la segunda $48$ litros en $6$ minutos y la tercera, $15$ litros en $3$ minutos, ¿en cuánto tiempo se llenará el estanque? \\
 \textbf{R.} \rule{2cm}{0.1mm}
 \item Un comerciante compró $30$ trajes a $\$ 2000$ cada uno. Vendió $20$ trajes a $\$ 1800$ cada uno. ¿En cuánto tiene que vender los restantes trajes para no perder en su inversión? \textbf{R.} \rule{2cm}{0.1mm}
 \item $50 - (6.31 + 14) = $ \textbf{R.} \rule{2cm}{0.1mm}
 \item $134 \times 0.873 =$ \textbf{R.} \rule{2cm}{0.1mm}
 \item $(0.5 + 0.76) \times 5 =$ \textbf{R.} \rule{2cm}{0.1mm}
 \item $0.243 \divisionsymbol 0.081 =$ \textbf{R.} \rule{2cm}{0.1mm}
 \item $17 \divisionsymbol 0.143 =$ \textbf{R.} \rule{2cm}{0.1mm}
 \item $0.64 \divisionsymbol 16 =$ \textbf{R.} \rule{2cm}{0.1mm}
\end{enumerate}

\end{document}