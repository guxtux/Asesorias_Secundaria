\input{preambulo_materiales}

\title{Los números primos \\ {\Large La criba de Eratóstenes}\vspace{-3ex}}
\author{}
\date{ }


\begin{document}
\maketitle
\fontsize{14}{14}\selectfont

\textbf{Definición:} Un número primo es aquel que solo es divisible entre sí mismo y el $1$.
\par
La criba de Eratóstenes nos permite identificar los números primos que hay entre $1$ y $100$, para ello sigamos el siguiente método:
\begin{enumerate}
\item Comienza marcado todos aquellos números que sean múltiplos de $2$.
\item Luego marca todos aquellos números que sean múltiplos de $3$.
\item Como ya marcaste los números que son múltiplos de $2$, por lo tanto, ya se marcaron los que son múltiplos de $4$, así que marca los números que son múltiplos de $5$.
\item Continua marcando los números que son múltiplos de los que vayas encontrando.
\item Al final tendremos una lista de los números primos que hay entre $1$ y $100$.
\end{enumerate}

\newpage

\begin{table}
\centering
\Large
\begin{tabular}{| c  | c | c  | c | c  | c | c  | c | c  | c |}
\hline
1 & 2 & 3 & 4 & 5 & 6 & 7 & 8 & 9 & 10 \\ \hline
11 & 12 & 13 & 14 & 15 & 16 & 17 & 18 & 19 & 20 \\ \hline
21 & 22 & 23 & 24 & 25 & 26 & 27 & 28 & 29 & 30 \\ \hline
31 & 32 & 33 & 34 & 35 & 36 & 37 & 38 & 39 & 40 \\ \hline
41 & 42 & 43 & 44 & 45 & 46 & 47 & 48 & 49 & 50 \\ \hline
51 & 52 & 53 & 54 & 55 & 56 & 57 & 58 & 59 & 60 \\ \hline
61 & 62 & 63 & 64 & 65 & 66 & 67 & 68 & 69 & 70 \\ \hline
71 & 72 & 73 & 74 & 75 & 76 & 77 & 78 & 79 & 80 \\ \hline
81 & 82 & 83 & 84 & 85 & 86 & 87 & 88 & 89 & 90 \\ \hline
91 & 92 & 93 & 94 & 95 & 96 & 97 & 98 & 99 & 100 \\ \hline
\end{tabular}
\end{table}
\vspace*{1em}
Anota los números que no quedaron marcados:
\\[0.5em]
\rule{14cm}{0.1mm}

\end{document}