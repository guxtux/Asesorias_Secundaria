\input{preambulo_materiales}

\title{Ejercicios de repaso \vspace{-2cm}}
\author{}
\date{ }

%--------------------------------------------------------------------
\newcounter{choice}
\renewcommand\thechoice{\Alph{choice})}
%\newcommand\choicelabel{\thechoice.}
\newcommand\choicelabel{\thechoice}

\newenvironment{choices}%
  {\list{\choicelabel}%
     {\usecounter{choice}\def\makelabel##1{\hss\llap{##1}}%
       \settowidth{\leftmargin}{W.\hskip\labelsep\hskip 2.5em}%
       \def\choice{%
         \item
       } % choice
       \labelwidth\leftmargin\advance\labelwidth-\labelsep
       \topsep=0pt
       \partopsep=0pt
     }%
  }%
  {\endlist}

\newenvironment{oneparchoices}%
  {%
    \setcounter{choice}{0}%
    \def\choice{%
      \refstepcounter{choice}%
      \ifnum\value{choice}>1\relax
        \penalty -50\hskip 1em plus 1em\relax
      \fi
      \choicelabel
      \nobreak\enskip
    }% choice
    % If we're continuing the paragraph containing the question,
    % then leave a bit of space before the first choice:
    \ifvmode\else\enskip\fi
    \ignorespaces
  }%
  {}
%----------------------------------------------------------

\begin{document}
\maketitle
\fontsize{14}{14}\selectfont

Resuelve cada una de las siguentes preguntas, toma en cuenta de que debes de elegir una sola respuesta. Se te pedirá en la revisión que muestres las operaciones que realizaste para llegar al resultado que te permitió elegir la respuesta.

\begin{enumerate}
\item ¿Cuál es el mínimo común múltiplo de $3, 5, 6$?
\begin{table}[H]
\large
\begin{tabular}{p{2cm} p{2cm} p{2cm} p{2cm} p{2cm}}
A) $60$ & B) $14$ & C) $30$ & D) $36$ & E) $18$ \\
\end{tabular}
\end{table}
\item ¿Cuál es la expresión en fracción del número mixto $15 \dfrac{2}{3}$?
\begin{table}[H]
    \large
\begin{tabular}{p{2cm} p{2cm} p{2cm} p{2cm} p{2cm}}
A) $\dfrac{31}{3}$ & B) $\dfrac{16}{3}$ & C) $\dfrac{23}{3}$ & D) $\dfrac{47}{3}$ & E) $\dfrac{51}{3}$ \\
\end{tabular}
\end{table}
\item ¿Cuál es la expresión en número mixto de la fracción $\dfrac{100}{11}$?
\begin{table}[H]
    \large
\begin{tabular}{p{2cm} p{2cm} p{2cm} p{2cm} p{2cm}}
A) $1 \dfrac{9}{11}$ & B) $9 \dfrac{5}{6}$ & C) $9 \dfrac{5}{11}$ & D) $10 \dfrac{1}{3}$ & E) $9 \dfrac{1}{11}$ \\
\end{tabular}
\end{table}
\item Reduce a su más simple expresión la fracción $\dfrac{28}{36}$?
\begin{table}[H]
    \large
\begin{tabular}{p{2cm} p{2cm} p{2cm} p{2cm} p{2cm}}
A) $\dfrac{7}{9}$ & B) $\dfrac{1}{2}$ & C) $\dfrac{2}{9}$ & D) $\dfrac{9}{16}$ & E) $\dfrac{2}{5}$ \\
\end{tabular}
\end{table}
\item Resuelve la siguiente suma de fracciones $1 \dfrac{1}{2} + 2 \dfrac{1}{3} + 1 \dfrac{1}{6}$
\begin{table}[H]
    \large
\begin{tabular}{p{2cm} p{2cm} p{2cm} p{2cm} p{2cm}}
A) $9$ & B) $5$ & C) $\dfrac{23}{3}$ & D) $2 \dfrac{1}{3}$ & E) $\dfrac{4}{5}$ \\
\end{tabular}
\end{table}
\end{enumerate}

\newpage

Ejercicios de tarea.

\begin{enumerate}
\item Cuatro hombres pesan $50 \dfrac{3}{4}$, $60 \dfrac{5}{8}$, $65 \dfrac{1}{12}$ y $80$ kilogramos respectivamente.
\\
¿Cuánto pesan entre los cuatro?
\begin{table}[H]
\large
\begin{tabular}{p{2.5cm} p{2.75cm} p{2.75cm} p{2.75cm} p{2.75cm}}
A) $256$ kg & B) $255 \dfrac{1}{24}$ kg & C) $256 \dfrac{1}{24}$ kg & D) $256 \dfrac{3}{12}$ kg & E) $256 \dfrac{3}{4}$ kg \\
\end{tabular}
\end{table}
\item Un agricultor vendió  $350 \dfrac{2}{3}$ kilos de papas, $750 \dfrac{5}{12}$ kilos de arroz, $125 \dfrac{3}{8}$ kilos de frijoles y $116 \dfrac{1}{18}$ kilos de café. respectivamente.
\\
¿Cuántos kilos de mercancía ha vendido?
\begin{choices}
\item $1342 \dfrac{37}{72}$ kg
\item $1234 \dfrac{37}{27}$ kg
\item $1340 \dfrac{35}{72}$ kg
\item $1342 \dfrac{1}{12}$ kg
\item $1340 \dfrac{37}{72}$ kg
\end{choices}
\item Simplifica la siguiente operación: $\dfrac{1}{2} - \dfrac{1}{8} - \dfrac{1}{40}$
\begin{table}[H]
\large
\begin{tabular}{p{2.5cm} p{2.75cm} p{2.75cm} p{2.75cm} p{2.75cm}}
A) $\dfrac{5}{20}$ & B) $\dfrac{11}{20}$ & C) $\dfrac{13}{20}$ & D) $\dfrac{19}{20}$ & E) $ \dfrac{7}{20}$ \\
\end{tabular}
\end{table}
\item Simplifica la siguiente operación: $11 \dfrac{3}{8} - 5 \dfrac{1}{24}$
\begin{table}[H]
\large
\begin{tabular}{p{2.5cm} p{2.75cm} p{2.75cm} p{2.75cm} p{2.75cm}}
A) $6 \dfrac{2}{3}$ & B) $6 \dfrac{1}{3}$ & C) $4 \dfrac{13}{120}$ & D) $8 \dfrac{1}{18}$ & E) $2 \dfrac{7}{9}$ \\
\end{tabular}
\end{table}
\item Simplifica la siguiente operación: $9 - 5 \dfrac{1}{6} + 4 \dfrac{1}{12}$
\begin{table}[H]
\large
\begin{tabular}{p{2.5cm} p{2.75cm} p{2.75cm} p{2.75cm} p{2.75cm}}
A) $1 \dfrac{17}{80}$ & B) $3 \dfrac{19}{40}$ & C) $6 \dfrac{14}{15}$ & D) $7 \dfrac{11}{12}$ & E) $2 \dfrac{47}{150}$ \\
\end{tabular}
\end{table}
\end{enumerate}
\end{document}