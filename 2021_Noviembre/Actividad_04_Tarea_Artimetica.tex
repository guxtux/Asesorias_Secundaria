\input{preambulo_materiales}

\title{Ejercicios de tarea \\ {\Large Artimética}\vspace{-3ex}}
\author{}
\date{ }


\begin{document}
\maketitle
\fontsize{14}{14}\selectfont

\textbf{Indicaciones: } Responde las siguientes preguntas:

\begin{enumerate}
\item $9 + 2 \times 3 =$ \hspace{0.3cm} \textbf{R.} \rule{3cm}{0.1mm}
\item $9 \times 3 - 4 \times 2 =$ \hspace{0.3cm} \textbf{R.} \rule{3cm}{0.1mm}
\item $9 + 6 \times 4 - 5 =$ \hspace{0.3cm} \textbf{R.} \rule{3cm}{0.1mm}
\item $(7 + 2) \times (5 + 4) =$ \hspace{0.3cm} \textbf{R.} \rule{3cm}{0.1mm}
\item Si en una división exacta el dividendo es $2940$ y el cociente $210$, ¿cuál es el divisor? \hspace{0.3cm} \textbf{R.} \rule{3cm}{0.1mm}
\item Se reparten $\$731$ entre varias personas por partes iguales, a cada una le tocan $\$43$. ¿Cuántas eran las personas?  \hspace{0.3cm} \textbf{R.} \rule{3cm}{0.1mm}
\item Repartí $243$ lápices entre $54$ personas y sobraron $27$ lápices. ¿Cuántos lápices le di a cada una? \hspace{0.3cm} \textbf{R.} \rule{3cm}{0.1mm}
\item Si $14$ libros cuestan $\$840$ ¿Cuánto costarían $9$ libros? \hspace{0.3cm} \textbf{R.} \rule{3cm}{0.1mm}
\item Si 19 sombreros cuestan $\$57$ ¿Cuántos sombreros se podrían comprar con $\$1080$  \hspace{0.3cm} \textbf{R.} \rule{3cm}{0.1mm}
\item Se reparten $84$ kg de víveres entre $3$ familias compuestas por $7$ personas cada una. ¿Cuántos kilos recibe cada persona? \hspace{0.3cm} \textbf{R.} \rule{3cm}{0.1mm}
\end{enumerate}


\end{document}