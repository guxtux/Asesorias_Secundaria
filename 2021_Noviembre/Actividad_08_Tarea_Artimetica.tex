\documentclass[12pt]{article}
\usepackage[utf8]{inputenc}
\usepackage[spanish]{babel}
\usepackage{amsmath}
\usepackage{amsthm}
\usepackage{hyperref}
\usepackage{graphicx}
\usepackage{color}
\usepackage{float}
\usepackage{multicol}
\usepackage{enumerate}
\usepackage{anyfontsize}
\usepackage{anysize}
\usepackage{tikz}
\usepackage{siunitx}
\usepackage{gensymb}
\usetikzlibrary{arrows.meta, positioning}
\setlength{\parskip}{1em}
\spanishdecimal{.}

\title{Ejercicios de Tarea \vspace{-2cm}}
\author{}
\date{ }


\begin{document}
\maketitle
\fontsize{14}{14}\selectfont

\textbf{Indicaciones: } Resuelve las siguientes operaciones:

\begin{enumerate}[label=\roman*)]
% \item Resuelve y simplifica las siguientes operaciones con fracciones:
% \begin{table}[H]
% \centering
% \def\arraystretch{2}
% \setlength{\tabcolsep}{18pt}
% \begin{tabular}{ | c | c | c | c | } 
% \hline
% $\dfrac{2}{3} + \dfrac{5}{6}$ & $\dfrac{5}{12} + \dfrac{7}{24}$ & $\dfrac{5}{8} + \dfrac{11}{64}$ & $\dfrac{7}{24} + \dfrac{11}{30}$ \\ \hline
%  & & &  \\ \hline  
% \end{tabular}
% \end{table}
% \item Resuelve y simplifica las siguientes fracciones:
% \begin{table}[H]
% \centering
% \def\arraystretch{2}
% \setlength{\tabcolsep}{18pt}
% \begin{tabular}{ | c | c | c | c | } 
% \hline
% $3 \dfrac{1}{4} + 5 \dfrac{3}{4}$ & $8 \dfrac{3}{7} + 6 \dfrac{5}{7}$ & $9 \dfrac{3}{5} + 4 \dfrac{1}{10}$ & $7 \dfrac{1}{8} + 3 \dfrac{5}{24}$ \\ \hline
%  & & & \\ \hline  
% \end{tabular}
% \end{table}
% \item Un hombre camina $4 \dfrac{1}{2}$ km el lunes, $8 \dfrac{2}{3}$ km el martes, $10$ km el miércoles y $\dfrac{5}{8}$ km el jueves. ¿Cuántos kilómetros ha recorrido en los cuatro días? \hspace{0.3cm} \textbf{R.} \rule{3cm}{0.1mm}
\item  Resuelve y simplifica las siguientes fracciones:
\begin{table}[H]
\centering
\def\arraystretch{2}
\setlength{\tabcolsep}{18pt}
\begin{tabular}{ | c | c | c | c | } 
\hline
$10 \dfrac{5}{6} - 2 \dfrac{7}{9}$ & $12 \dfrac{2}{3} - 7 \dfrac{1}{11}$ & $7 + \dfrac{8}{7}$ & $18 + \dfrac{6}{5}$ \\ \hline
 & & & \\ \hline  
\end{tabular}
\end{table}
\item  Resuelve y simplifica las siguientes fracciones:
\begin{table}[H]
\centering
\def\arraystretch{2}
\setlength{\tabcolsep}{18pt}
\begin{tabular}{ | c | c | c | c | } 
\hline
$\dfrac{2}{3} + \dfrac{5}{6} - \dfrac{1}{12}$ & $\dfrac{3}{4} - \dfrac{5}{8} - \dfrac{7}{12}$ & $\dfrac{7}{12} + \dfrac{5}{9} - \dfrac{4}{24}$ & $\dfrac{11}{15} - \dfrac{7}{30} + \dfrac{3}{10}$ \\ \hline
 & & & \\ \hline  
\end{tabular}
\end{table}
\item Resuelve y simplifica las siguientes fracciones:
\begin{table}[H]
\centering
\def\arraystretch{2}
\setlength{\tabcolsep}{15pt}
\begin{tabular}{ | c | c | c | c | } 
\hline
$4 \dfrac{1}{2} + \left( \dfrac{1}{6} + \dfrac{1}{12}\right)$ & $7 \dfrac{1}{4} - \left( \dfrac{3}{5} - \dfrac{1}{6}\right)$ & $3 \dfrac{5}{8} - \left( 2 \dfrac{3}{4} + \dfrac{1}{8} \right)$ & $50 - \left( 6 - \dfrac{1}{5} \right)$ \\ \hline
 & & & \\ \hline  
\end{tabular}
\end{table}
\item Una calle tiene $50 \dfrac{2}{3}$ metros de longitud y otra calle tiene $45 \dfrac{5}{8}$ metros. ¿Cuántos metros tienen las dos juntas? ¿Cuánto le falta a cada una de ellas para tener $80$ metros? \hspace{0.3cm} \textbf{R.} \rule{3cm}{0.1mm}
\end{enumerate}
\end{document}