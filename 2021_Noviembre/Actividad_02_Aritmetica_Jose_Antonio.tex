\documentclass[12pt]{article}
\usepackage[utf8]{inputenc}
\usepackage[spanish]{babel}
\usepackage{amsmath}
\usepackage{amsthm}
\usepackage{hyperref}
\usepackage{graphicx}
\usepackage{color}
\usepackage{float}
\usepackage{multicol}
\usepackage{enumerate}
\usepackage{anyfontsize}
\usepackage{anysize}
\usepackage{tikz}
\usepackage{siunitx}
\usepackage{gensymb}
\usetikzlibrary{arrows.meta, positioning}
\setlength{\parskip}{1em}
\spanishdecimal{.}

\title{Ejercicios de repaso - José Antonio \vspace{-2cm}}
\author{}
\date{ }

%--------------------------------------------------------------------
\newcounter{choice}
\renewcommand\thechoice{\Alph{choice})}
%\newcommand\choicelabel{\thechoice.}
\newcommand\choicelabel{\thechoice}

\newenvironment{choices}%
  {\list{\choicelabel}%
     {\usecounter{choice}\def\makelabel##1{\hss\llap{##1}}%
       \settowidth{\leftmargin}{W.\hskip\labelsep\hskip 2.5em}%
       \def\choice{%
         \item
       } % choice
       \labelwidth\leftmargin\advance\labelwidth-\labelsep
       \topsep=0pt
       \partopsep=0pt
     }%
  }%
  {\endlist}

\newenvironment{oneparchoices}%
  {%
    \setcounter{choice}{0}%
    \def\choice{%
      \refstepcounter{choice}%
      \ifnum\value{choice}>1\relax
        \penalty -50\hskip 1em plus 1em\relax
      \fi
      \choicelabel
      \nobreak\enskip
    }% choice
    % If we're continuing the paragraph containing the question,
    % then leave a bit of space before the first choice:
    \ifvmode\else\enskip\fi
    \ignorespaces
  }%
  {}
%----------------------------------------------------------

\begin{document}
\maketitle
\fontsize{14}{14}\selectfont

Escribe en notación decimal:
\begin{enumerate}[label=\alph*)]
\item $115$ diezmilésimas. \hspace{0.3cm} \textbf{R.} \rule{3cm}{0.1mm}
\item $318$ cienmilésimas. \hspace{0.3cm} \textbf{R.} \rule{3cm}{0.1mm}
\item $133346$ cienmilésimas. \hspace{0.3cm} \textbf{R.} \rule{3cm}{0.1mm}
\item $7$ unidades con $19$ milésimas. \hspace{0.3cm} \textbf{R.} \rule{3cm}{0.1mm}
\item $9$ unidades con $9$ diezmilésimas. \hspace{0.3cm} \textbf{R.} \rule{3cm}{0.1mm}
\item $6$ unidades con $16$ cienmilésimas. c
\end{enumerate}

Lee en voz alta las siguientes cantidades:
\begin{enumerate}[label=\alph*)]
\item $0.8$
\item $0.15$
\item $0.09$
\item $0.0015$
\item $1.015$
\item $1.15678$
\item $2.000016$
\end{enumerate}

\newpage

Ejercicios de tarea.
\\
\\
Cuando se tienen operaciones sin paréntesis, primero se resuelven las divisiones, luego las multiplicaciones y al final las sumas y restas, conforme se presenten de izquierda a derecha.
\\
\\
Resuelve:
\begin{enumerate}
\item $8 + 6 \divisionsymbol 3 =$ \hspace{0.3cm} \textbf{R.} \rule{3cm}{0.1mm}
\item $15 \divisionsymbol 5 - 2 =$ \hspace{0.3cm} \textbf{R.} \rule{3cm}{0.1mm}
\item $12 \divisionsymbol 4 \times 3 + 5 =$ \hspace{0.3cm} \textbf{R.} \rule{3cm}{0.1mm}
\item $14 \divisionsymbol 3 \times 4 \divisionsymbol 2 \times 6 =$ \hspace{0.3cm} \textbf{R.} \rule{3cm}{0.1mm}
\item $15 + 6 \divisionsymbol 3 - 4 \divisionsymbol 2 + 4 =$ \hspace{0.3cm} \textbf{R.} \rule{3cm}{0.1mm}
\end{enumerate}
\end{document}