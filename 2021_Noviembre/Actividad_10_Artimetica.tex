\documentclass[12pt]{article}
\usepackage[utf8]{inputenc}
\usepackage[spanish]{babel}
\usepackage{amsmath}
\usepackage{amsthm}
\usepackage{hyperref}
\usepackage{graphicx}
\usepackage{color}
\usepackage{float}
\usepackage{multicol}
\usepackage{enumerate}
\usepackage{anyfontsize}
\usepackage{anysize}
\usepackage{tikz}
\usepackage{siunitx}
\usepackage{gensymb}
\usetikzlibrary{arrows.meta, positioning}
\setlength{\parskip}{1em}
\spanishdecimal{.}

\title{Ejercicios de tarea \vspace{-2cm}}
\author{}
\date{ }


\begin{document}
\maketitle
\fontsize{14}{14}\selectfont

Responde cada pregunta:
\begin{enumerate}
\item Pedro ha estudiado $3 \, \dfrac{2}{3}$ horas, Enrique $5 \, \dfrac{3}{4}$ horas y Juan $6$ horas, ¿Cuánto tiempo han estudiado los tres juntos? \hspace{0.3cm} \textbf{R.} \rule{3cm}{0.1mm}
\item Un campesino ha cosechado $2500$ kilos de papas, $250 \, \dfrac{1}{8}$ de trigo y $180 \,\dfrac{2}{9}$ de arroz. ¿Cuántos kilos ha cosechado en conjunto? \hspace{0.3cm} \textbf{R.} \rule{3cm}{0.1mm}
\item Tres varillas tienen cada una la siguiente longitud: la primera $8 \,\dfrac{2}{5}$ metros de largo, la segunda $10 \,\dfrac{3}{10}$ metros y la tercera $14 \,\dfrac{1}{20}$ metros, si juntamos las tres varillas una tras otra ¿Cuál es la longitud total? \\ 
\hspace{0.3cm} \textbf{R.} \rule{3cm}{0.1mm}
\item El lunes ahorré $\$ 20 \, \dfrac{3}{4}$, el martes $\$ 50 \, \dfrac{5}{8}$, el miércoles $\$ 70 \, \dfrac{1}{12}$ y el jueves $\$ 10 \, \dfrac{1}{24}$. ¿Cuánto he ahorrado? \hspace{0.3cm} \textbf{R.} \rule{3cm}{0.1mm}
\item Pedro tiene $22 \dfrac{2}{9}$ años, Juan $6 \dfrac{1}{3}$ años más que Pedro y la edad de Matías es la de Juan y Pedro juntos. ¿Cuántos años suman las tres edades? \textbf{R.} \rule{3cm}{0.1mm}
\item Resuelve las siguientes operaciones con fracciones:
\begin{enumerate}[label=\alph*)]
\item $\dfrac{3}{5} - \dfrac{1}{10} =$ \textbf{R.} \rule{3cm}{0.1mm} 
\item $\dfrac{11}{12} - \dfrac{7}{16} =$ \textbf{R.} \rule{3cm}{0.1mm} 
\item $\dfrac{93}{120} - \dfrac{83}{150} =$ \textbf{R.} \rule{3cm}{0.1mm} 
\item $\dfrac{1}{2} - \dfrac{1}{8} - \dfrac{1}{40} =$ \textbf{R.} \rule{3cm}{0.1mm} 
\item $\dfrac{7}{35} - \dfrac{1}{100} - \dfrac{11}{1000} =$ \textbf{R.} \rule{3cm}{0.1mm} 
\end{enumerate}
\item Resuelve las siguientes operaciones con fracciones:
\begin{enumerate}[label=\alph*)]
\item $6 \, \dfrac{5}{6} - 3 \, \dfrac{1}{6} =$ \textbf{R.} \rule{3cm}{0.1mm} 
\item $9 \, \dfrac{1}{6} - 7 \, \dfrac{2}{3} =$ \textbf{R.} \rule{3cm}{0.1mm} 
\item $7 \, \dfrac{3}{5} - 4 \, \dfrac{3}{10} =$ \textbf{R.} \rule{3cm}{0.1mm} 
\item $8 \, \dfrac{1}{8} - 2 \, \dfrac{3}{4} - \dfrac{1}{40} =$ \textbf{R.} \rule{3cm}{0.1mm} 
\item $115 \dfrac{5}{27} - 101 \, \dfrac{7}{9} =$ \textbf{R.} \rule{3cm}{0.1mm} 
\end{enumerate}
\item Simplifica las siguientes operaciones con fracciones:
\begin{enumerate}[label=\alph*)]
\item $\dfrac{2}{3} + \dfrac{5}{6} - \dfrac{1}{12}=$ \textbf{R.} \rule{3cm}{0.1mm} 
\item $\dfrac{3}{4} - \dfrac{5}{8} + \dfrac{7}{12} =$ \textbf{R.} \rule{3cm}{0.1mm} 
\item $\dfrac{1}{6} - \dfrac{1}{7} + \dfrac{1}{12} - \dfrac{1}{14} =$ \textbf{R.} \rule{3cm}{0.1mm} 
\item $\dfrac{6}{9} + \dfrac{15}{25} - \dfrac{8}{15} =$ \textbf{R.} \rule{3cm}{0.1mm} 
\item $\dfrac{1}{50} - \dfrac{2}{75} + \dfrac{7}{150} - \dfrac{1}{180} =$ \textbf{R.} \rule{3cm}{0.1mm} 
\end{enumerate}
\item Debo $\$ 183$ pesos y pago $\$42 \, \dfrac{2}{7}$. ¿Cuánto me falta por pagar? \\ \textbf{R.} \rule{3cm}{0.1mm} 
\item Tengo $\$ 6 \, \dfrac{3}{5}$ ¿Cuánto necesito para tener $\$ 8 \, \dfrac{1}{6}$ \textbf{R.} \rule{3cm}{0.1mm} 
\item Un hombre vende $\dfrac{1}{8}$ de su finca, alquila $\dfrac{1}{8}$ y lo restante lo cultiva. ¿Qué porción de la finca cultiva? \textbf{R.} \rule{3cm}{0.1mm}  
\item Un hombre vende $\dfrac{1}{3}$ de su finca, alquila $\dfrac{1}{8}$ y lo restante lo cultiva. ¿Qué porción de la finca cultiva? \textbf{R.} \rule{3cm}{0.1mm}  
\end{enumerate}
\end{document}