\documentclass[12pt]{article}
\usepackage[utf8]{inputenc}
\usepackage[spanish]{babel}
\usepackage{amsmath}
\usepackage{amsthm}
\usepackage{hyperref}
\usepackage{graphicx}
\usepackage{color}
\usepackage{float}
\usepackage{multicol}
\usepackage{enumerate}
\usepackage{anyfontsize}
\usepackage{anysize}
\usepackage{tikz}
\usepackage{siunitx}
\usepackage{gensymb}
\usetikzlibrary{arrows.meta, positioning}
\setlength{\parskip}{1em}
\spanishdecimal{.}

\title{Ejercicios de repaso - María \vspace{-2cm}}
\author{}
\date{ }


\begin{document}
\maketitle
\fontsize{14}{14}\selectfont

\textbf{Indicaciones: } Resuelve las siguientes operaciones:

\begin{enumerate}[label=\roman*)]
\item En una obra se trabajan $8$ horas al día y se hacen $5$ metros en una hora. ¿Cuántos días se necesitarán para hacer $360$?  \hspace{0.3cm} \textbf{R.} \rule{3cm}{0.1mm}
\item De las siguientes fracciones, indica con color rojo aquellos que son mayores que la unidad, con color azul aquellos que son menores a la unidad y con lápiz aquellos son iguales a la unidad:
\begin{align*}
\dfrac{5}{7}, \dfrac{16}{9}, \dfrac{15}{15}, \dfrac{31}{96},\dfrac{114}{113}, \dfrac{19}{14}  
\end{align*}
\item ¿Cuánto hay que añadir a cada una de las siguientes fracciones para que sean iguales a la unidad?
\begin{align*}
\dfrac{8}{11}, \dfrac{14}{25}, \dfrac{18}{19}, \dfrac{106}{231}  
\end{align*}
\hspace{0.3cm} \textbf{R.} \rule{3cm}{0.1mm}
\item Convierte a fracciones los siguientes números mixtos:
\begin{table}[H]
\centering
\def\arraystretch{2}
\setlength{\tabcolsep}{18pt}
\begin{tabular}{ | c | c | c | c | c | } 
\hline
$15 \dfrac{3}{8}$ & $20 \dfrac{3}{19}$ & $42 \dfrac{7}{25}$ & $5 \dfrac{3}{106}$ & $90 \dfrac{19}{37}$ \\ \hline
 & & & & \\ \hline  
\end{tabular}
\end{table}
\item Simplifica las siguientes fracciones
\begin{table}[H]
\centering
\def\arraystretch{2}
\setlength{\tabcolsep}{18pt}
\begin{tabular}{ | c | c | c | c | c | } 
\hline
$\dfrac{2 \times 6}{6 \times 8}$ & $\dfrac{10 \times 7}{7 \times 5}$ & $\dfrac{9 \times 8}{18 \times 65}$ & $\dfrac{2 \times 6}{14 \times 8}$ & $\dfrac{3 \times 2 \times 5}{6 \times 4 \times 10}$ \\ \hline
 & & & & \\ \hline  
\end{tabular}
\end{table}
\end{enumerate}


\end{document}