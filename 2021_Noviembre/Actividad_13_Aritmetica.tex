\input{preambulo_materiales}

\title{Ejercicios de repaso \vspace{-2cm}}
\author{}
\date{ }

%--------------------------------------------------------------------
\newcounter{choice}
\renewcommand\thechoice{\Alph{choice})}
%\newcommand\choicelabel{\thechoice.}
\newcommand\choicelabel{\thechoice}

\newenvironment{choices}%
  {\list{\choicelabel}%
     {\usecounter{choice}\def\makelabel##1{\hss\llap{##1}}%
       \settowidth{\leftmargin}{W.\hskip\labelsep\hskip 2.5em}%
       \def\choice{%
         \item
       } % choice
       \labelwidth\leftmargin\advance\labelwidth-\labelsep
       \topsep=0pt
       \partopsep=0pt
     }%
  }%
  {\endlist}

\newenvironment{oneparchoices}%
  {%
    \setcounter{choice}{0}%
    \def\choice{%
      \refstepcounter{choice}%
      \ifnum\value{choice}>1\relax
        \penalty -50\hskip 1em plus 1em\relax
      \fi
      \choicelabel
      \nobreak\enskip
    }% choice
    % If we're continuing the paragraph containing the question,
    % then leave a bit of space before the first choice:
    \ifvmode\else\enskip\fi
    \ignorespaces
  }%
  {}
%----------------------------------------------------------

\begin{document}
\maketitle
\fontsize{14}{14}\selectfont

Escribe en el espacio la respuesta a las siguientes preguntas:
\begin{enumerate}[label=\alph*)]
\item ¿Cuál de las siguientes fracciones es la mayor, cuál la menor? Explica tu respuesta.  $\dfrac{7}{10}, \dfrac{7}{10}, \dfrac{7}{19}, \dfrac{7}{23}$ \hspace{0.3cm} \textbf{R.} \rule{3cm}{0.1mm}
\item ¿Cuál de las siguientes fracciones es la mayor, cuál la menor? Explica tu respuesta.  $\dfrac{5}{6}, \dfrac{11}{6}, \dfrac{13}{6}, \dfrac{19}{6}$ \hspace{0.3cm} \textbf{R.} \rule{3cm}{0.1mm}
\item Escribe de menor a mayor las siguientes fracciones $\dfrac{3}{5}, \dfrac{11}{13},\dfrac{5}{7}$ 
\\[0.3em]
\textbf{R.} \rule{3cm}{0.1mm}
\item Escribe de mayor a menor las siguientes fracciones $\dfrac{21}{17}, \dfrac{9}{5},\dfrac{7}{3}$ 
\\
\textbf{R.} \rule{3cm}{0.1mm}
\item Escribe en orden, de menor a mayor los números decimales de cada inciso.
\begin{enumerate}[label=\roman*)]
\item $1.02, 1.002, 1.015, 1.11$
\item $6.606, 6.66, 6.5999, 6.509$
\item $0.0078, 0.708, 0.078, 0.0087$
\end{enumerate}
\end{enumerate}s

\end{document}