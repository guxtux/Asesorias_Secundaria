\input{preambulo_materiales}

\title{Ejercicios de repaso \vspace{-2cm}}
\author{}
\date{ }


\begin{document}
\maketitle
\fontsize{14}{14}\selectfont

\begin{enumerate}
\item Halla los enteros contenidos en:
\begin{enumerate}[label=\alph*)]
\item $\dfrac{12}{3}$ \hspace{0.3cm} \textbf{R.} \rule{3cm}{0.1mm}
\item $\dfrac{21}{7}$ \hspace{0.3cm} \textbf{R.} \rule{3cm}{0.1mm}
\item $\dfrac{108}{12}$ \hspace{0.3cm} \textbf{R.} \rule{3cm}{0.1mm}
\item $\dfrac{63}{5}$ \hspace{0.3cm} \textbf{R.} \rule{3cm}{0.1mm}
\item $\dfrac{95}{18}$ \hspace{0.3cm} \textbf{R.} \rule{3cm}{0.1mm}
\end{enumerate}
\item Reduce las siguientes cantidades:
\begin{enumerate}[label=\alph*)]
\item $2$ a tercios. \hspace{0.3cm} \textbf{R.} \rule{3cm}{0.1mm}
\item $4$ a cuartos. \hspace{0.3cm} \textbf{R.} \rule{3cm}{0.1mm}
\item $15$ a onceavos. \hspace{0.3cm} \textbf{R.} \rule{3cm}{0.1mm}
\item $26$ a treceavos. \hspace{0.3cm} \textbf{R.} \rule{3cm}{0.1mm}
\item $43$ a 51avos. \hspace{0.3cm} \textbf{R.} \rule{3cm}{0.1mm}
\end{enumerate}
\item Reduce a su más simple expresión:
\begin{enumerate}[label=\alph*)]
\item $\dfrac{28}{36}$ \hspace{0.3cm} \textbf{R.} \rule{3cm}{0.1mm}
\item $\dfrac{306}{1452}$ \hspace{0.3cm} \textbf{R.} \rule{3cm}{0.1mm}
%\item $\dfrac{54}{96}$ \hspace{0.3cm} \textbf{R.} \rule{3cm}{0.1mm}
%\item $\dfrac{99}{165}$ \hspace{0.3cm} \textbf{R.} \rule{3cm}{0.1mm}
\item $\dfrac{98}{105}$ \hspace{0.3cm} \textbf{R.} \rule{3cm}{0.1mm}
\item $\dfrac{4459}{4802}$ \hspace{0.3cm} \textbf{R.} \rule{3cm}{0.1mm}
\item $\dfrac{1690}{3549}$ \hspace{0.3cm} \textbf{R.} \rule{3cm}{0.1mm}
\end{enumerate}
\newpage
\item Reduce al mínimo común denominador:
\begin{enumerate}[label=\alph*)]
\item $\dfrac{1}{2}, \dfrac{1}{4}$. \hspace{0.3cm} \textbf{R.} \rule{3cm}{0.1mm}
\item $\dfrac{1}{3}, \dfrac{1}{6}$. \hspace{0.3cm} \textbf{R.} \rule{3cm}{0.1mm}
\item $\dfrac{1}{3}, \dfrac{2}{9}$. \hspace{0.3cm} \textbf{R.} \rule{3cm}{0.1mm}
\item $\dfrac{1}{5}, \dfrac{1}{10}, \dfrac{1}{12}$. \hspace{0.3cm} \textbf{R.} \rule{3cm}{0.1mm}
\item $\dfrac{1}{5}, \dfrac{3}{10}, \dfrac{7}{20}, \dfrac{11}{40}$. \hspace{0.3cm} \textbf{R.} \rule{3cm}{0.1mm}
\end{enumerate}
\item Simplifica las siguientes fracciones:
\begin{enumerate}[label=\alph*)]
\item $7 + \dfrac{8}{7}$. \hspace{0.3cm} \textbf{R.} \rule{3cm}{0.1mm}
\item $\dfrac{14}{12} + 60$. \hspace{0.3cm} \textbf{R.} \rule{3cm}{0.1mm}
\item $\dfrac{7}{45} + 4 + \dfrac{11}{60} + \dfrac{7}{36}$. \hspace{0.3cm} \textbf{R.} \rule{3cm}{0.1mm}
\item $4 + \dfrac{7}{48} + 8 \dfrac{1}{57} + \dfrac{1}{114}$. \hspace{0.3cm} \textbf{R.} \rule{3cm}{0.1mm}
\item $14 + 5 \dfrac{2}{3}$. \hspace{0.3cm} \textbf{R.} \rule{3cm}{0.1mm}
\end{enumerate}
\end{enumerate}
\end{document}