\input{preambulo_materiales}

\title{Ejercicios de repaso - José Antonio \vspace{-2cm}}
\author{}
\date{ }

%--------------------------------------------------------------------
\newcounter{choice}
\renewcommand\thechoice{\Alph{choice})}
%\newcommand\choicelabel{\thechoice.}
\newcommand\choicelabel{\thechoice}

\newenvironment{choices}%
  {\list{\choicelabel}%
     {\usecounter{choice}\def\makelabel##1{\hss\llap{##1}}%
       \settowidth{\leftmargin}{W.\hskip\labelsep\hskip 2.5em}%
       \def\choice{%
         \item
       } % choice
       \labelwidth\leftmargin\advance\labelwidth-\labelsep
       \topsep=0pt
       \partopsep=0pt
     }%
  }%
  {\endlist}

\newenvironment{oneparchoices}%
  {%
    \setcounter{choice}{0}%
    \def\choice{%
      \refstepcounter{choice}%
      \ifnum\value{choice}>1\relax
        \penalty -50\hskip 1em plus 1em\relax
      \fi
      \choicelabel
      \nobreak\enskip
    }% choice
    % If we're continuing the paragraph containing the question,
    % then leave a bit of space before the first choice:
    \ifvmode\else\enskip\fi
    \ignorespaces
  }%
  {}
%----------------------------------------------------------

\begin{document}
\maketitle
\fontsize{14}{14}\selectfont

Efectúa las siguientes operaciones:

\begin{enumerate}[label=\alph*)]
\item $0.3 +0.8 +3.15 = $
\item $0.19 + 3.81+ 0.723+ 0.131 = $
\item $0.39 - 0.184 = $
\item $0.76 + 31.893 - 14 = $
\item $0.17 \times 0.83 = $
\item $56 \divisionsymbol 0.114 =$
\item $56.03 \divisionsymbol 19 =$
\end{enumerate}

\newpage

Ejercicios de tarea.

\begin{enumerate}
\item $8 \divisionsymbol 0.512 =$
\item $12 \divisionsymbol 0.003 =$
\item $17 \divisionsymbol 0.143 =$
\item $14 \times 0.08 =$
\item $16.84 \times 14.35 =$
\item $(0.5 + 0.76) \times 5 =$
\item $(8.35 + 6.003 + 0.01) \times 0.7 =$
\item $(0.75 - 0.3) \times 5 =$
\item Pedro tiene $\$ 5.64$, Juan $\$2.37$ más que Pedro y Enrique $\$1.15$ más que Juan. ¿Cuánto tienen entre los tres? \hspace{0.3cm} \textbf{R.} \rule{3cm}{0.1mm}
\item Inicié la semana el lunes con $\$ 14.25$, el martes cobré $\$ 16.89$, el miércoles cobré $\$ 97$ y el jueves pagué $\$ 56.07$. ¿Cuánto me queda?
\\
\hspace{0.3cm} \textbf{R.} \rule{3cm}{0.1mm}
\end{enumerate}
\end{document}