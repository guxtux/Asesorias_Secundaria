\input{preambulo_materiales}

\title{Ejercicios de Tarea \\ {\Large Artimética}\vspace{-3ex}}
\author{}
\date{ }


\begin{document}
\maketitle
\fontsize{14}{14}\selectfont

\textbf{Indicaciones: } Resuelve las siguientes operaciones:

\begin{enumerate}[label=\arabic*)]
\item $(0.5 + 0.76) \times 5 =$ \hspace{0.3cm} \textbf{R.} \rule{3cm}{0.1mm}
\item $(8.35 + 6.003 + 0.01) \times 0.7 =$ \hspace{0.3cm} \textbf{R.} \rule{3cm}{0.1mm}
\item $(0.75 - 0.3) \times 5 =$ \hspace{0.3cm} \textbf{R.} \rule{3cm}{0.1mm}
\item $(0.978 - 0.0013) \times 8.01 =$ \hspace{0.3cm} \textbf{R.} \rule{3cm}{0.1mm}
\item $(14 - 0.1) \times 31 =$ \hspace{0.3cm} \textbf{R.} \rule{3cm}{0.1mm}
\item $9 + 6 \times 4 - 5 =$ \hspace{0.3cm} \textbf{R.} \rule{3cm}{0.1mm}
\item $5 \times 7 - 3 + 8 \times 2 =$ \hspace{0.3cm} \textbf{R.} \rule{3cm}{0.1mm}
\item $(6 + 5 + 4) \times 3 =$ \hspace{0.3cm} \textbf{R.} \rule{3cm}{0.1mm}
\item $(8 + 5 + 3) \times (6  - 4) =$ \hspace{0.3cm} \textbf{R.} \rule{3cm}{0.1mm}
\item $8 + 6 \divisionsymbol 3 =$ \hspace{0.3cm} \textbf{R.} \rule{3cm}{0.1mm}
\item $15 \divisionsymbol 5 - 2 =$ \hspace{0.3cm} \textbf{R.} \rule{3cm}{0.1mm}
\item $12 \divisionsymbol 4 \times 3 + 5 =$ \hspace{0.3cm} \textbf{R.} \rule{3cm}{0.1mm}
\item Pedro tiene $\$564.37$, Juan $\$237.56$ más que Pedro y Enrique $\$115.63$ más que Juan. ¿Cuánto tienen entre los tres? \hspace{0.3cm} \textbf{R.} \rule{3cm}{0.1mm}
\item Un hombre se compra un traje, un sombrero, un bastón y una billetera. La billetera le costó $\$375.82$, el sombrero le ha costado el doble de lo que le costó la billetera, el bastón costó $\$178.39$ más que el sombrero, y el traje $5$ veces más que la billetera. ¿Cuánto pagó por todo? \\
\hspace{0.3cm} \textbf{R.} \rule{3cm}{0.1mm}
\item Un comerciante hace un pedido por $3000$ kg de mercancías y se lo envían en cuatro entregas. En la primera le mandan $72.45$ kg, en la segunda $40$ kg más que en la primera, en la tercera le envían lo que en la primera y segunda, en la cuarta entrega le envían lo restante. ¿Cuántos kg le enviaron en la última partida? \hspace{0.3cm} \textbf{R.} \rule{3cm}{0.1mm}
\item Si $14$ libros cuestan $\$840$. ¿Cuánto costarían $9$ libros? \hspace{0.2cm} \textbf{R.} \rule{3cm}{0.1mm}
\end{enumerate}

\end{document}