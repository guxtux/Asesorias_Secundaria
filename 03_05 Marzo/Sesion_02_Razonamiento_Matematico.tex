\input{preambulo_presentacion_Berlin_beaver}
\usepackage{pifont}
\newcommand{\cmark}{\ding{51}}%
\newcommand{\xmark}{\ding{55}}%
\definecolor{cadmiumgreen}{rgb}{0.0, 0.42, 0.24}
\makeatletter
\setbeamertemplate{footline}
{
  \leavevmode%
  \hbox{%
  \begin{beamercolorbox}[wd=.333333\paperwidth,ht=2.25ex,dp=1ex,center]{section in foot}%
    \usebeamerfont{section in foot} \insertsection
  \end{beamercolorbox}%
  \begin{beamercolorbox}[wd=.333333\paperwidth,ht=2.25ex,dp=1ex,center]{subsection in foot}%
    \usebeamerfont{subsection in foot}  \insertsubsection
  \end{beamercolorbox}%
  \begin{beamercolorbox}[wd=.333333\paperwidth,ht=2.25ex,dp=1ex,right]{date in head/foot}%
    \usebeamerfont{date in head/foot} \insertshortdate{} \hspace*{2em}
    \insertframenumber{} / \inserttotalframenumber \hspace*{2ex} 
  \end{beamercolorbox}}%
  \vskip0pt%
}
\makeatother
%--------------------------------------------------------------------
%--------------------------------------------------------------------
\newcounter{choice}
\renewcommand\thechoice{\Alph{choice})}
%\newcommand\choicelabel{\thechoice.}
\newcommand\choicelabel{\thechoice}

\newenvironment{choices}%
  {\list{\choicelabel}%
     {\usecounter{choice}\def\makelabel##1{\hss\llap{##1}}%
       \settowidth{\leftmargin}{W.\hskip\labelsep\hskip 2.5em}%
       \def\choice{%
         \item
       } % choice
       \labelwidth\leftmargin\advance\labelwidth-\labelsep
       \topsep=0pt
       \partopsep=0pt
     }%
  }%
  {\endlist}

\newenvironment{oneparchoices}%
  {%
    \setcounter{choice}{0}%
    \def\choice{%
      \refstepcounter{choice}%
      \ifnum\value{choice}>1\relax
        \penalty -50\hskip 1em plus 1em\relax
      \fi
      \choicelabel
      \nobreak\enskip
    }% choice
    % If we're continuing the paragraph containing the question,
    % then leave a bit of space before the first choice:
    \ifvmode\else\enskip\fi
    \ignorespaces
  }%
  {}
%----------------------------------------------------------
%----------------------------------------------------------

\setbeamertemplate{navigation symbols}{}
\date{17 de marzo de 2021}
\title{Sesión 2. Razonamiento matemático}
\subtitle{Asesoría}
\begin{document}
\maketitle
\fontsize{14}{14}\selectfont
\spanishdecimal{.}
\section*{Contenido}
\frame[allowframebreaks]{\tableofcontents[currentsection, hideallsubsections]}
\section{Habilidad matemática}
\frame{\tableofcontents[currentsection, hideothersubsections]}

\subsection{Ejercicio 1}

\begin{frame}
\frametitle{Ejercicio 1}
¿Qué número sigue a esta sucesión numérica?
\begin{align*}
0.1, \, 0.02, \, 0.003, \, 0.0004, \, \_\_ , \, \_\_
\end{align*}
\begin{choices}
\choice $0.5, 0.6$ \\
\choice $0.005, 0.00006$ \\
\choice $0.00005, 0.00006$ \\
\choice $0.00005, 0.000006$ \\
\choice $0.50000, 0.600000$ \\
\end{choices}
\end{frame}
\begin{frame}
\frametitle{Tipo de sucesión}
Revisamos como primer punto el tipo de sucesión que se nos presenta:
\begin{align*}
0.1, \, 0.02, \, 0.003, \, 0.0004, \, \_\_ , \, \_\_
\end{align*}
\pause
¿Qué tipo de sucesión tenemos?
\pause
\setbeamercolor{item projected}{bg=blue!70!black,fg=yellow}
\setbeamertemplate{enumerate items}[circle]
\begin{enumerate}[<+->]
\item Creciente
\item Decreciente
\end{enumerate}
\end{frame}
\begin{frame}
\frametitle{Recordando operación con decimales}
Sabemos que los números con decimales los podemos representar como:
\begin{align*}
0.1, \, 0.02, \, 0.003, \, 0.0004, \, \_\_ , \, \_\_
\end{align*}
\pause
\begin{tikzpicture}[overlay, font=\footnotesize]
    \draw [thick, ->] (2.9, 0) -- (2.9, 1);
    \node at (2.9, -0.4) {Décimas}; \pause
    \draw [thick, ->] (4, -0.7) -- (4, 1);
    \node at (4, -1) {Centésimas}; \pause
    \draw [thick, ->] (5, -1.4) -- (5, 1);
    \node at (5, -1.7) {Milésimas}; \pause
    \draw [thick, ->] (6.8, -0.7) -- (6.4, -0.7) -- (6.4, 1);
    \node at (7.9, -0.7) {Diezmilésimas}; 
\end{tikzpicture}
\end{frame}
\begin{frame}
\frametitle{Sucesión decreciente}
Tenemos entonces una sucesión decreciente.
\\
\bigskip
\pause
Además, hemos identificado el patrón con el que va cambiando el siguiente elemento de la sucesión.
\\
\bigskip
\pause
Los valores que siguen en la sucesión corresponden a \textbf{cienmilésimo} y a un \textbf{millonésimo}.
\end{frame}
\begin{frame}
\frametitle{Eliminando incisos}
Con la información que ya encontramos, podemos eliminar algunos incisos:
\begin{align*}
0.1, \, 0.02, \, 0.03, \, 0.004, \, \_\_ , \, \_\_
\end{align*}
\begin{choices}
\choice $0.5, 0.6$ \\
\choice $0.005, 0.00006$ \\
\choice $0.00005, 0.00006$ \\
\choice $0.00005, 0.000006$ \\
\choice $0.50000, 0.600000$ \\
\end{choices}
\begin{tikzpicture}[overlay, thick, font=\small]
    \draw [<-, red] (4, 4.1) -- (6, 4.1);
    \node at (7.6, 4.1) {Décima, Décima}; \pause
    \node [text = red] at (0.9, 4.1) {\xmark}; \pause
    \draw [<-, red] (5.2, 3.3) -- (6.6, 3.3);
    \node at (8.7, 3.3) {Milésima, Diezmilésima}; \pause
    \node [text = red] at (0.9, 3.3) {\xmark}; \pause
    \draw [<-, red] (5.6, 2.5) -- (6.3, 2.5);
    \node at (8.7, 2.5) {Cienmilésima, Cienmilésima}; \pause
    \node [text = red] at (0.9, 2.5) {\xmark}; \pause
    \draw [<-, red] (5.8, 1.7) -- (6.5, 1.7);
    \node at (8.7, 1.7) {Cienmilésima, Millonésima}; \pause
    \node [text = cadmiumgreen] at (0.9, 1.7) {\cmark}; \pause
    \node [text = red] at (0.9, 0.8) {\xmark};
\end{tikzpicture}
\end{frame}

\subsection{Ejercicio 2}

\begin{frame}
\frametitle{Ejercicio 2}
¿Qué par de números continuan la sucesión numérica?
\begin{align*}
1, \, 9, \, 2, \, 9, \, \_\_ , \, \_\_, \, 4, \, 9, \, 5, \, 9
\end{align*}
\begin{choices}
\choice $8, 7$ \\
\choice $8, 5$ \\
\choice $3, 9$ \\
\choice $3, 3$ \\
\choice $8, 3$ \\
\end{choices}
\pause
\begin{tikzpicture}[overlay]
    \node [text = cadmiumgreen] at (1, 2.7) {\cmark};
\end{tikzpicture}
\end{frame}

\subsection{Resolviendo ejercicios}

\begin{frame}
\frametitle{Ejercicio 3}
¿Qué par de números continuan la sucesión numérica?
\begin{align*}
\frac{1}{2}, \, \frac{2}{4}, \, \frac{3}{6}, \, \frac{4}{8}, \, \_\_ , \, \_\_, \, \frac{7}{14}, \, \frac{8}{16}
\end{align*}
\begin{choices}
\choice $\frac{5}{8}, \frac{6}{8}$ \\
\choice $\frac{5}{8}, \frac{6}{9}$ \\
\choice $\frac{5}{10}, \frac{6}{10}$ \\
\choice $\frac{5}{9}, \frac{6}{11}$ \\
\choice $\frac{5}{10}, \frac{6}{12}$ \\
\end{choices}
\pause
\begin{tikzpicture}[overlay]
    \node [text = cadmiumgreen] at (1, 1) {\cmark};
\end{tikzpicture}
\end{frame}
\begin{frame}
\frametitle{Ejercicio 4}
¿Qué par de números continuan la sucesión numérica?
\begin{align*}
1, \, 5, \, 2, \, 10, \, \_\_ , \, \_\_, \, 4, \, 20, \, 5
\end{align*}
\begin{choices}
\choice $3, 15$ \\
\choice $11, 12$ \\
\choice $11, 3$ \\
\choice $5, 6$ \\
\choice $2, 15$ \\
\end{choices}
\pause
\begin{tikzpicture}[overlay]
    \node [text = cadmiumgreen] at (1, 4.4) {\cmark};
\end{tikzpicture}
\end{frame}
\begin{frame}
\frametitle{Indicaciones}
En los siguientes ejercicios se presentan  secuencias de números y letras que están incompletas.
\\
\bigskip
\pause
Selecciona la opción que mejor completa la secuencia de números y letras que se presenta.
\end{frame}
\begin{frame}
\frametitle{Ejercicio 5}
La sucesión es:
\begin{align*}
2 \, a, \,  6 \, a \, b, \,  24 \, a \, b \, c, \, \_
\end{align*}
\begin{choices}
\choice $12 \, a$ \\
\choice $24 \, a \, b$ \\
\choice $72 \, a \, b \, c$ \\
\choice $120 \, a \, b \, c \, d$ \\
\end{choices}
\pause
\begin{tikzpicture}[overlay]
    \node [text = cadmiumgreen] at (1, 1) {\cmark};
\end{tikzpicture}
\end{frame}
\begin{frame}
\frametitle{Ejercicio 6}
La sucesión es:
\begin{align*}
2 \, y^{2}, \,  4 \, y^{3}, \, 8 \, y^{4},  \, \_
\end{align*}
\begin{choices}
\choice $12 \, y^{4}$ \\
\choice $14 \, y^{6}$ \\
\choice $16 \, y^{5}$ \\
\choice $18 \, y^{6}$
\end{choices}
\pause
\begin{tikzpicture}[overlay]
    \node [text = cadmiumgreen] at (1, 1.8) {\cmark};
\end{tikzpicture}
\end{frame}
\begin{frame}
\frametitle{Ejercicio 7}
La sucesión es:
\begin{align*}
60, \, 120, \, 40, \, 80, \, 20, \, \_\_ \, , \, \_\_
\end{align*}
\begin{choices}
\choice $40, \, 0$ \\
\choice $40, \, 8$ \\
\choice $60, \, 12$ \\
\choice $60, \, 20$
\end{choices}
\pause
\begin{tikzpicture}[overlay]
    \node [text = cadmiumgreen] at (1, 3.5) {\cmark};
\end{tikzpicture}
\end{frame}
\begin{frame}
\frametitle{Ejercicio 8}
La sucesión es:
\begin{align*}
1 \, B, \, 2 \, A, \, 3 \, D, \, 4 \, C, \, 5 \, F, \, \_\_ \, , \, \_\_
\end{align*}
\begin{choices}
\choice $6 \, E, \, 7 \, H$ \\
\choice $6 \, E, \, 7 \, G$ \\
\choice $6 \, G, \, 7 \, H$ \\
\choice $6 \, H, \, 7 \, G$
\end{choices}
\pause
\begin{tikzpicture}[overlay]
    \node [text = cadmiumgreen] at (1, 2.55) {\cmark};
\end{tikzpicture}
\end{frame}

\section{Batería de ejercicios}
\frame{\tableofcontents[currentsection, hideothersubsections]}
\subsection{Solución con tiempo}

\begin{frame}
\frametitle{Ejercicios de repaso}
Para repasar lo que hemos visto, ahora se te presenta una batería de ejercicios.
\\
\bigskip
\pause
Lee cuidadosamente las preguntas y elige entre las opciones, aquella que contenga la respuesta correcta.
\end{frame}
\begin{frame}
\frametitle{Ejercicio 9}
$0, \, 8, \, 2, \, 6, \, \ldots$, los números que siguen son:
\begin{choices}
\choice $4, \, 10$ \\
\choice $8, \, 0$ \\
\choice $4, \, 4$ \\
\choice $10, \, 4$ \\
\choice $4, \, 6$
\end{choices}
\pause
\begin{tikzpicture}[overlay]
    \node [text = cadmiumgreen] at (1, 2.55) {\cmark};
\end{tikzpicture}
\end{frame}
\begin{frame}
\frametitle{Ejercicio 10}
$1, \, 2, \, 6, \, \ldots, \, 120$, el número que falta es:
\begin{choices}
\choice $36$ \\
\choice $24$ \\
\choice $12$ \\
\choice $60$ \\
\choice $30$
\end{choices}
\pause
\begin{tikzpicture}[overlay]
    \node [text = cadmiumgreen] at (1, 3.5) {\cmark};
\end{tikzpicture}
\end{frame}
\begin{frame}
\frametitle{Ejercicio 11}
$\dfrac{2}{4}, \, \dfrac{4}{9}, \, \dfrac{6}{16}, \, \dfrac{8}{25}, \, \ldots$, la fracción que falta es:
\\
\bigskip
\begin{choices}
\choice $\frac{10}{36}$ \\
\choice $\frac{6}{16}$ \\
\choice $\frac{9}{49}$ \\
\choice $\frac{12}{64}$ \\
\choice $\frac{16}{36}$
\end{choices}
\pause
\begin{tikzpicture}[overlay]
    \node [text = cadmiumgreen] at (1, 4.3) {\cmark};
\end{tikzpicture}
\end{frame}
\begin{frame}
\frametitle{Ejercicio 12}
¿Qué tipo de patrón numérico se obseva en la siguente serie artimética $1, \, 4, \, 9, \, 16, \, 25, \ldots$, 
\begin{choices}
\choice Aumenta consecutivamente. \\
\choice Disminuye radicalmente. \\
\choice Aumenta exponencialmente. \\
\choice Aumenta radicalmente. \\
\choice Disminuye exponencialmente.
\end{choices}
\pause
\begin{tikzpicture}[overlay]
    \node [text = cadmiumgreen] at (1, 4.3) {\cmark};
\end{tikzpicture}
\end{frame}
\begin{frame}
\frametitle{Ejercicio 13}
Observa la siguente serie de números:
\begin{align*}
2, \, 6, \, 10, \, 14, \, 18, \ldots
\end{align*}
¿Qué número ocupará el lugar 32?
\begin{choices}
\choice $126$ \\
\choice $128$ \\
\choice $78$ \\
\choice $218$ \\
\choice $130$
\end{choices}
\pause
\begin{tikzpicture}[overlay]
    \node [text = cadmiumgreen] at (1, 4.1) {\cmark};
\end{tikzpicture}
\end{frame}
\begin{frame}
\frametitle{¿Por qué es cierto?}
Este ejercicio presenta una patrón particular entre los números de la sucesión.
\\
\bigskip
\pause
Presentemos los elementos de otra manera y estudiemos ese patrón.
\end{frame}
\begin{frame}
\frametitle{La sucesión de números}
La sucesión la podemos ver como:
\begin{eqnarray*}
&{}& 2, \, 6, \, 10, \, 14, \, 18, \ldots \\[0.5em] \pause
&{}& 2(1), \, 2(3), \, 2(5), \, 2(7), \, 2(9), \ldots
\end{eqnarray*}
\pause
\begin{tikzpicture}[overlay]
  \node at (1.8, 0.6) {$n =$}; \pause
  \node at (3, 0.6) {$1$}; \pause
  \node at (4.2, 0.6) {$2$}; \pause
  \node at (5.3, 0.6) {$3$}; \pause
  \node at (6.4, 0.6) {$4$}; \pause
  \node at (7.6, 0.6) {$5$}; \pause
  \node at (8.5, 0.6) {$\ldots$};
\end{tikzpicture}
\pause
\\
\bigskip
Hay que determinar una expresión que relacione el valor que multiplica a $2$ y el valor de $n$.
\end{frame}
\begin{frame}
\frametitle{Relación entre $n$ y $2$}
Vemos que los valores entre paréntesis son números impares, entonces veamos si con la siguiente expresión los podemos recuperar:
\pause
\begin{align*}
(2 \, n - 1)
\end{align*}
\end{frame}
\begin{frame}
\frametitle{Obtiendo algunos valores}
Si hacemos una tabla para verificar que efectivamente recuperamos los números impares, será más fácil comprobar el resultado:
\end{frame}
\begin{frame}
\frametitle{Obtiendo algunos valores}
\fontsize{12}{12}\selectfont
\begin{table}
\centering
\begin{tabular}{c | c}
$n$ & $(2 \, n - 1)$ \\ \hline
$1$ & $1$ \\ \hline
$2$ & $3$ \\ \hline
$3$ & $5$ \\ \hline
$4$ & $7$ \\ \hline
$5$ & $9$ \\ \hline
$\ldots$ & $\ldots$ \\ \hline
\end{tabular}
\end{table}
\end{frame}
\begin{frame}
\frametitle{El valor para la respuesta exacta}
Como el enunciado pide que digamos qué número ocuparla el lugar $n = 32$, entonces podemos usar la expresión que encontramos y multiplicamos por $2$:
\pause
\begin{eqnarray*}
2 \, (2 (32) - 1) &=& 2 \, (64 - 1) = \\[0.5em] \pause
&=& 2 \, (63) = \\[0.5em]
&=& 126
\end{eqnarray*}
\pause
Por lo que la respuesta del inciso \textbf{A)} es la correcta.
\end{frame}
\end{document}