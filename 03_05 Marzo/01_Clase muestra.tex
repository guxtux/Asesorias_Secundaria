\input{preambulo_presentacion_Berlin_beaver}
\makeatletter
% \setbeamercolor{section in foot}{bg=gray!30, fg=black!90!orange}
% \setbeamercolor{subsection in foot}{bg=blue!30!yellow, fg=red}
% \setbeamercolor{date in foot}{bg=black, fg=white}
\setbeamertemplate{footline}
{
  \leavevmode%
  \hbox{%
  \begin{beamercolorbox}[wd=.333333\paperwidth,ht=2.25ex,dp=1ex,center]{section in foot}%
    \usebeamerfont{section in foot} \insertsection
  \end{beamercolorbox}%
  \begin{beamercolorbox}[wd=.333333\paperwidth,ht=2.25ex,dp=1ex,center]{subsection in foot}%
    \usebeamerfont{subsection in foot}  \insertsubsection
  \end{beamercolorbox}%
  \begin{beamercolorbox}[wd=.333333\paperwidth,ht=2.25ex,dp=1ex,right]{date in head/foot}%
    \usebeamerfont{date in head/foot} \insertshortdate{} \hspace*{2em}
    \insertframenumber{} / \inserttotalframenumber \hspace*{2ex} 
  \end{beamercolorbox}}%
  \vskip0pt%
}
\makeatother
\setbeamertemplate{navigation symbols}{}
\date{5 de marzo de 2021}
\title{Clase muestra}
\subtitle{Asesoría}
\begin{document}
\maketitle
\fontsize{14}{14}\selectfont
\spanishdecimal{.}
\section*{Contenido}
\frame[allowframebreaks]{\tableofcontents[currentsection, hideallsubsections]}
\section{Temas}
\frame[allowframebreaks]{\tableofcontents[currentsection, hideothersubsections]}

\subsection{Habilidad matemática}
\begin{frame}
\frametitle{Habilidades}
En este apartado se busca que el estudiante desarrolle  distintas habilidades:
\setbeamercolor{item projected}{bg=blue!70!black,fg=yellow}
\setbeamertemplate{enumerate items}[circle]
\begin{enumerate}[<+->]
\item Razonamiento.
\item Abstracción.
\item Lógica.
\item Solución de problemas.
\end{enumerate}
\end{frame}
\begin{frame}
\frametitle{Combinación de habilidades}
Al plantear un problema, se espara que el estudiante ocupe sus habilidades más el conocimiento matemático para resolver el ejercicio.
\\
\bigskip
\pause
Debido apoyarse tanto con la artimética y el álgebra.
\end{frame}
\begin{frame}
\frametitle{Ejemplo}
¿Qué número sigue en esta sucesión numérica?
\begin{align*}
4, 8, 16, 20, \ldots
\end{align*}

\setbeamercolor{item projected}{bg=blue!70!black,fg=yellow}
\setbeamertemplate{enumerate items}[circle]
\begin{enumerate}
\item $18$
\item $24$
\item $16$
\item $20$
\item $28$
\end{enumerate}
\end{frame}
\subsection{Matemáticas}

\subsection{Estadística}

\subsection{Física}
\end{document}