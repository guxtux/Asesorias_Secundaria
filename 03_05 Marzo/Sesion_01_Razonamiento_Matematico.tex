%\RequirePackage{currfile}
\documentclass[12pt]{beamer}
\usepackage[utf8]{inputenc}
\usepackage[spanish]{babel}
\usepackage{standalone}
\usepackage{color}
\usepackage{siunitx}
\usepackage{hyperref}
\usepackage[outdir=./]{epstopdf}
%\hypersetup{colorlinks,linkcolor=,urlcolor=blue}
%\hypersetup{colorlinks,urlcolor=blue}
\usepackage{anyfontsize}
\usepackage{lmodern}
\usepackage{xcolor,soul}
\usepackage{etoolbox}
\usepackage{amsmath}
\usepackage{amsthm}
\usepackage{mathtools}
\usepackage{tcolorbox}
\usepackage{physics}
\usepackage{multicol}
\usepackage{bookmark}
\usepackage{longtable}
\usepackage{listings}
\usepackage{cancel}
\usepackage{wrapfig}
\usepackage{empheq}
\usepackage{graphicx}
\usepackage{tikz}
\usetikzlibrary{calc, patterns, matrix, backgrounds, decorations,shapes, arrows.meta}
\usepackage[autostyle,spanish=mexican]{csquotes}
\usepackage[os=win]{menukeys}
\usepackage{pifont}
\usepackage{pbox}
\usepackage{bm}
\usepackage{caption}
\captionsetup{font=scriptsize,labelfont=scriptsize}
%\usepackage[sfdefault]{roboto}  %% Option 'sfdefault' only if the base font of the document is to be sans serif

%Sección de definición de colores
\definecolor{ao}{rgb}{0.0, 0.5, 0.0}
\definecolor{bisque}{rgb}{1.0, 0.89, 0.77}
\definecolor{amber}{rgb}{1.0, 0.75, 0.0}
\definecolor{armygreen}{rgb}{0.29, 0.33, 0.13}
\definecolor{alizarin}{rgb}{0.82, 0.1, 0.26}
\definecolor{cadetblue}{rgb}{0.37, 0.62, 0.63}
\definecolor{deepblue}{rgb}{0,0,0.5}
\definecolor{brown}{rgb}{0.59, 0.29, 0.0}
\definecolor{OliveGreen}{rgb}{0,0.25,0}
\definecolor{mycolor}{rgb}{0.122, 0.435, 0.698}

\newcommand*{\boxcolor}{orange}
\makeatletter
\newcommand{\boxedcolor}[1]{\textcolor{\boxcolor}{%
\tikz[baseline={([yshift=-1ex]current bounding box.center)}] \node [rectangle, minimum width=1ex, thick, rounded corners,draw] {\normalcolor\m@th$\displaystyle#1$};}}
 \makeatother

 \newcommand*\widefbox[1]{\fbox{\hspace{2em}#1\hspace{2em}}}

\newtcbox{\mybox}{on line,
  colframe=mycolor,colback=mycolor!10!white,
  boxrule=0.5pt,arc=4pt,boxsep=0pt,left=6pt,right=6pt,top=6pt,bottom=6pt}

\usefonttheme[onlymath]{serif}
%Sección de definición de nuevos comandos

\newcommand*{\TitleParbox}[1]{\parbox[c]{1.75cm}{\raggedright #1}}%
\newcommand{\python}{\texttt{python}}
\newcommand{\textoazul}[1]{\textcolor{blue}{#1}}
\newcommand{\azulfuerte}[1]{\textcolor{blue}{\textbf{#1}}}
\newcommand{\funcionazul}[1]{\textcolor{blue}{\textbf{\texttt{#1}}}}
\newcommand{\ptilde}[1]{\ensuremath{{#1}^{\prime}}}
\newcommand{\stilde}[1]{\ensuremath{{#1}^{\prime \prime}}}
\newcommand{\ttilde}[1]{\ensuremath{{#1}^{\prime \prime \prime}}}
\newcommand{\ntilde}[2]{\ensuremath{{#1}^{(#2)}}}
\renewcommand{\arraystretch}{1.5}

\newcounter{saveenumi}
\newcommand{\seti}{\setcounter{saveenumi}{\value{enumi}}}
\newcommand{\conti}{\setcounter{enumi}{\value{saveenumi}}}
\renewcommand{\rmdefault}{cmr}% cmr = Computer Modern Roman

\linespread{1.5}

\usefonttheme{professionalfonts}
%\usefonttheme{serif}
\DeclareGraphicsExtensions{.pdf,.png,.jpg}

%Sección para el tema de beamer, con el theme, usercolortheme y sección de footers
\mode<presentation>
{
  \usetheme{Berlin}
  
  %\useoutertheme{infolines}
  \useoutertheme{default}
  \usecolortheme{beaver}
  \setbeamercovered{invisible}
  % or whatever (possibly just delete it)
  \setbeamertemplate{section in toc}[sections numbered]
  \setbeamertemplate{subsection in toc}[subsections numbered]
  \setbeamertemplate{subsection in toc}{\leavevmode\leftskip=3.2em\rlap{\hskip-2em\inserttocsectionnumber.\inserttocsubsectionnumber}\inserttocsubsection\par}
  \setbeamercolor{section in toc}{fg=blue}
  \setbeamercolor{subsection in toc}{fg=blue}
  \setbeamercolor{frametitle}{fg=blue}
  \setbeamertemplate{caption}[numbered]

  \setbeamertemplate{footline}
  \beamertemplatenavigationsymbolsempty
  \setbeamertemplate{headline}{}
}

\makeatletter
\setbeamercolor{section in foot}{bg=gray!30, fg=black!90!orange}
\setbeamercolor{subsection in foot}{bg=blue!30!yellow, fg=red}
\setbeamertemplate{footline}
{
  \leavevmode%
  \hbox{%
  \begin{beamercolorbox}[wd=.333333\paperwidth,ht=2.25ex,dp=1ex,center]{section in foot}%
    \usebeamerfont{section in foot} \insertsection
  \end{beamercolorbox}}%
  \begin{beamercolorbox}[wd=.333333\paperwidth,ht=2.25ex,dp=1ex,center]{subsection in foot}%
    \usebeamerfont{subsection in foot}  \insertsubsection
  \end{beamercolorbox}%
  \begin{beamercolorbox}[wd=.333333\paperwidth,ht=2.25ex,dp=1ex,right]{date in head/foot}%
    \usebeamerfont{date in head/foot} \insertshortdate{} \hspace*{2em}
    \insertframenumber{} / \inserttotalframenumber \hspace*{2ex} 
  \end{beamercolorbox}}%
  \vskip0pt%
\makeatother  

\makeatletter
\patchcmd{\beamer@sectionintoc}
  {\vfill}
  {\vskip\itemsep}
  {}
  {}
\makeatother


\usepackage{pifont}
\newcommand{\cmark}{\ding{51}}%
\newcommand{\xmark}{\ding{55}}%
\definecolor{cadmiumgreen}{rgb}{0.0, 0.42, 0.24}
\makeatletter
\setbeamertemplate{footline}
{
  \leavevmode%
  \hbox{%
  \begin{beamercolorbox}[wd=.333333\paperwidth,ht=2.25ex,dp=1ex,center]{section in foot}%
    \usebeamerfont{section in foot} \insertsection
  \end{beamercolorbox}%
  \begin{beamercolorbox}[wd=.333333\paperwidth,ht=2.25ex,dp=1ex,center]{subsection in foot}%
    \usebeamerfont{subsection in foot}  \insertsubsection
  \end{beamercolorbox}%
  \begin{beamercolorbox}[wd=.333333\paperwidth,ht=2.25ex,dp=1ex,right]{date in head/foot}%
    \usebeamerfont{date in head/foot} \insertshortdate{} \hspace*{2em}
    \insertframenumber{} / \inserttotalframenumber \hspace*{2ex} 
  \end{beamercolorbox}}%
  \vskip0pt%
}
\makeatother
%--------------------------------------------------------------------
%--------------------------------------------------------------------
\newcounter{choice}
\renewcommand\thechoice{\Alph{choice})}
%\newcommand\choicelabel{\thechoice.}
\newcommand\choicelabel{\thechoice}

\newenvironment{choices}%
  {\list{\choicelabel}%
     {\usecounter{choice}\def\makelabel##1{\hss\llap{##1}}%
       \settowidth{\leftmargin}{W.\hskip\labelsep\hskip 2.5em}%
       \def\choice{%
         \item
       } % choice
       \labelwidth\leftmargin\advance\labelwidth-\labelsep
       \topsep=0pt
       \partopsep=0pt
     }%
  }%
  {\endlist}

\newenvironment{oneparchoices}%
  {%
    \setcounter{choice}{0}%
    \def\choice{%
      \refstepcounter{choice}%
      \ifnum\value{choice}>1\relax
        \penalty -50\hskip 1em plus 1em\relax
      \fi
      \choicelabel
      \nobreak\enskip
    }% choice
    % If we're continuing the paragraph containing the question,
    % then leave a bit of space before the first choice:
    \ifvmode\else\enskip\fi
    \ignorespaces
  }%
  {}
%----------------------------------------------------------
%----------------------------------------------------------

\setbeamertemplate{navigation symbols}{}
\date{16 de marzo de 2021}
\title{Sesión 1. Razonamiento matemático}
\subtitle{Asesoría}
\begin{document}
\maketitle
\fontsize{14}{14}\selectfont
\spanishdecimal{.}
\section*{Contenido}
\frame[allowframebreaks]{\tableofcontents[currentsection, hideallsubsections]}
\section{Habilidad matemática}
\frame[allowframebreaks]{\tableofcontents[currentsection, hideothersubsections]}

\subsection{Introducción}
\begin{frame}
\frametitle{Habilidades}
La prueba de \emph{Razonamiento o Habilidad Matemática}, se ha diseñado para medir habilidades que se relacionan con el trabajo.
\\
\bigskip
\pause
La habilidad de aplicar las matemáticas en situaciones nuevas y diferentes, es de gran importancia para el siguiente nivel de bachillerato.
\end{frame}
\begin{frame}
\frametitle{Áreas para desarrollar la habilidad}
Los ejercicios de razonamiento matemático miden la habilidad para procesar, analizar y utilizar información en la \emph{Aritmética, el Álgebra y la Geometría}.
\end{frame}
\begin{frame}
\frametitle{¿Qué es la habilidad matemática?}
La \emph{Habilidad Matemática} es aquella en que el alumno es capaz de \emph{comprender conceptos, proponer y efectuar algoritmos y desarrollar aplicaciones a través de la resolución de problemas}.
\end{frame}
\begin{frame}
\frametitle{Enfoque para el desarrollo}
En estas se consideran tres aspectos.
\setbeamercolor{item projected}{bg=blue!70!black,fg=yellow}
\setbeamertemplate{enumerate items}[circle]
\begin{enumerate}[<+->]
\item En \textcolor{blue}{Aritmética}: en operaciones fundamentales como suma, resta, multiplicación, división, potenciación y radicación, con números enteros y racionales, cálculos de porcentajes, proporciones y promedios, series numéricas y comparación de cantidades.
\seti
\end{enumerate}
\end{frame}
\begin{frame}
\frametitle{Enfoque para el desarrollo}
\setbeamercolor{item projected}{bg=blue!70!black,fg=yellow}
\setbeamertemplate{enumerate items}[circle]
\begin{enumerate}[<+->]
\conti
\item En \textcolor{blue}{Álgebra}: operaciones fundamentales con literales, simplificaciones de expresiones algebraicas, simbolización de expresiones, operaciones con potencias y raíces, factorización, ecuaciones y funciones lineales y cuadráticas.
\seti
\end{enumerate}
\end{frame}
\begin{frame}
\frametitle{Enfoque para el desarrollo}
\setbeamercolor{item projected}{bg=blue!70!black,fg=yellow}
\setbeamertemplate{enumerate items}[circle]
\begin{enumerate}[<+->]
\conti
\item En \textcolor{blue}{Geometría}: perímetros y áreas de figuras geométricas, propiedades de los triángulos (principales teoremas), propiedades de rectas paralelas y perpendiculares y Teorema de Pitágoras.
\seti
\end{enumerate}
\end{frame}
\begin{frame}
\frametitle{Enfoque para el desarrollo}
\setbeamercolor{item projected}{bg=blue!70!black,fg=yellow}
\setbeamertemplate{enumerate items}[circle]
\begin{enumerate}[<+->]
\conti
\item \textcolor{blue}{Sucesiones numéricas}: Serie de términos formados de acuerdo con una ley.
\item \textcolor{blue}{Series Espaciales}: Son figuras o trazos que siguen reglas o patrones determinados.
\item \textcolor{blue}{Imaginación Espacial}: Hay que echar a andar nuestra imaginación al $100\%$, ya que se presentan trazos, recortes y dobleces sin tener que hacerlo físicamente.
\end{enumerate}
\end{frame}

\subsection{Ejercicio 1}

\begin{frame}
\frametitle{Ejercicio 1}
¿Qué número sigue a esta sucesión numérica?
\begin{align*}
4, \, 8, \, 12, \, 16, \, 20, \ldots
\end{align*}
\begin{choices}
\choice $18$ \\
\choice $24$ \\
\choice $16$ \\
\choice $20$ \\
\choice $28$ \\
\end{choices}
\end{frame}
\begin{frame}
\frametitle{Punto relevante}
De las cinco posibles respuestas, \textbf{solo una es la correcta}.
\\
\bigskip
\pause
Hay preguntas a las que se les llama \emph{confusoras}, es decir, aparentan tener sentido con el enunciado, pero al hacer un análisis de las mismas, nos daremos cuenta de que no son correctas.
\end{frame}
\begin{frame}
\frametitle{Primera estrategia}
Si hacemos una revisión de la información que nos da el enunciado y la secuencia, veremos que hay elementos que destacan:
\pause
\begin{align*}
4, \, 8, \, 12, \, 16, \, 20, \ldots
\end{align*}
\setbeamercolor{item projected}{bg=blue!70!black,fg=yellow}
\setbeamertemplate{enumerate items}[circle]
\begin{enumerate}[<+->]
\item Los números de la sucesión son pares.
\item Hay un orden ascendente.
\end{enumerate}
\end{frame}
\begin{frame}[fragile]
\frametitle{Segunda estrategia}
Veamos que en las posibles respuestas, hay valores que ya están presentes en la sucesión, por lo que estos valores los debemos de descartar como respuesta:
\begin{align*}
4, \, 8, \, 12, \, 16, \, 20, \ldots
\end{align*}
\begin{choices}
\choice $18$ \\
\choice $24$ \\
\choice $16$ \\
\choice $20$ \\
\choice $28$ \\
\end{choices}
\pause
\begin{tikzpicture}[overlay]
  \draw [fill=red, opacity=0.25] (1.2, 2) rectangle (2.6, 2.7);
  \draw [fill=red, opacity=0.25] (5.3, 4.4) rectangle (5.9, 5.1);
  \draw [thick, red] (2.8, 2.4) -- (5.5, 2.4) -- (5.5, 4.2); \pause
  \node [text = red] at (0.8, 2.35) {\xmark};
  \pause
  \draw [fill=blue, opacity=0.25] (1.2, 1.3) rectangle (2.6, 1.9);
  \draw [fill=blue, opacity=0.25] (6, 4.4) rectangle (6.7, 5.1);
  \draw [thick, blue] (2.8, 1.6) -- (6.3, 1.6) -- (6.3, 4.2);
  \pause
  \node [text = red] at (0.8, 1.6) {\xmark};
\end{tikzpicture}
\end{frame}
\begin{frame}
\frametitle{Segunda estrategia}
Vemos que el inciso \textbf{A)} indica un valor, pero como sabemos que la sucesión es creciente, el valor de $18$ no podría ser el siguiente.
\end{frame}
\begin{frame}
\frametitle{Segunda estrategia}
\vspace*{-2cm}
\begin{align*}
  4, \, 8, \, 12, \, 16, \, 20, \ldots
\end{align*}
\begin{choices}
\choice $18$ \\
\choice $24$ \\
\choice $16$ \\
\choice $20$ \\
\choice $28$ \\
\end{choices}
\pause
\begin{tikzpicture}[overlay]
  \draw [fill=red, opacity=0.25] (1.2, 3.9) rectangle (2.6, 4.6);
  \draw [fill=red, opacity=0.25] (5.2, 5.1) rectangle (6.8, 5.7);
  \draw [thick, red] (2.8, 4.3) -- (6, 4.3) -- (6, 5);
  \node [text = red] at (0.8, 2.5) {\xmark};
  \node [text = red] at (0.8, 1.7) {\xmark};
  \pause
  \node [text = red] at (0.8, 4.25) {\xmark};
\end{tikzpicture}

Por lo que lo descartamos.
\end{frame}
\begin{frame}
\frametitle{Segunda estrategia}
Quedan dos incisos, uno es el correcto:
\begin{align*}
  4, \, 8, \, 12, \, 16, \, 20, \ldots
\end{align*}
\begin{choices}
\choice $18$ \\
\choice $24$ \\
\choice $16$ \\
\choice $20$ \\
\choice $28$ \\
\end{choices}
\pause
\begin{tikzpicture}[overlay]
  \node [text = red] at (0.8, 2.5) {\xmark};
  \node [text = red] at (0.8, 1.7) {\xmark};
  \node [text = red] at (0.8, 4.25) {\xmark};
\end{tikzpicture}
\end{frame}
\begin{frame}
\frametitle{Determinando patrones}
El siguiente paso es identificar un \emph{posible patrón} entre los valores de la sucesión.
\\
\bigskip
\pause
Podemos elegir el recorrido entre:
\pause
\setbeamercolor{item projected}{bg=blue!70!black,fg=yellow}
\setbeamertemplate{enumerate items}[circle]
\begin{enumerate}[<+->]
\item De izquierda a derecha.
\item De derecha a izquierda.
\end{enumerate}
\end{frame}
\begin{frame}
\frametitle{Determinando patrones}
De izquierda a derecha:
\begin{eqnarray*}
&4& \, (+ 4 = ) \, 8, \\[0.25em] \pause
&8& \pause \, (+ 4 =) \, 12, \\[0.25em] \pause
&12& \pause (+ 4 =) \, 16, \\[0.25em] \pause
&16& \pause (+ 4 = ) \, 20, \\[0.25em] \pause
&20& \pause (+ 4 = ) \, \pause 24 \\[0.25em] \pause
&24& \pause (+4 = ) \, \pause 28
\end{eqnarray*}
\end{frame}
\begin{frame}
\frametitle{Solución}
Hemos identificado el patrón de la sucesión: a cada término se le suma $4$, por lo que el siguiente término es $24$, el inciso \textbf{E)} con la respuesta $28$ aunque sería parte de la sucesión, no es el término siguiente de los que muestra el enunciado.
\end{frame}
\begin{frame}
\frametitle{Solución}
Entonces la solución es:
\begin{align*}
  4, \, 8, \, 12, \, 16, \, 20, \ldots
\end{align*}
\begin{choices}
\choice $18$ \\
\choice $24$ \\
\choice $16$ \\
\choice $20$ \\
\choice $28$ \\
\end{choices}
\pause
\begin{tikzpicture}[overlay]
  \node [text = cadmiumgreen] at (0.8, 3.5) {\cmark};
\end{tikzpicture}
\end{frame}

\subsection{Ejercicio 2}
\begin{frame}
\frametitle{Ejercicio 2}
De las parejas de números propuestas, elige la que corresponda en la sucesión:
\begin{align*}
1, \, 9, \, 2, \, 8, \, 3, \, 7, \, \_\_ \, , \, \_\_
\end{align*}
\begin{choices}
\choice $8, \, 2$ \\
\choice $2, \, 9$ \\
\choice $8, \, 10$ \\
\choice $6, \, 4$ \\
\choice $4, \, 6$ \\
\end{choices}
\end{frame}
\begin{frame}
\frametitle{Revisando el enunciado}
El enunciado nos señala que ahora se tienen \emph{parejas de números}, por lo que el primer paso es identificar las parejas mediante paréntesis:
\pause
\begin{align*}
(1, \, 9), \, (2, \, 8), \, (3, \, 7), \, (\_\_ \, , \, \_\_)
\end{align*}
\end{frame}
\begin{frame}
\frametitle{Analizando las parejas de números}
El análisis se deberá de realizar a cada número de la pareja, con el respectivo de las otras parejas, para identificar un patrón.
\end{frame}
\begin{frame}
\frametitle{Revisando los valores}
\setbeamercolor{item projected}{bg=blue!70!black,fg=yellow}
\setbeamertemplate{enumerate items}[circle]
\begin{enumerate}[<+->]
\item Cada pareja incluye valores enteros (sin decimales)
\item Hagamos una revisión con el primer valor de cada pareja de números.
\item Para luego hacer una revisión con el segundo valor de cada pareja de números.
\end{enumerate}
\end{frame}
\begin{frame}
\frametitle{Primer valor de la pareja de números}
Tenemos entonces que con el primer valor de la pareja de números:
\begin{align*}
(1, \, 9), \, (2, \, 8), \, (3, \, 7), \, (\_\_ \, , \, \_\_)
\end{align*}
\begin{tikzpicture}[overlay]
  \draw [fill=blue, opacity=0.25] (2.6, 1) rectangle (3, 1.6);
  \draw [fill=blue, opacity=0.25] (4.1, 1) rectangle (4.5, 1.6);
  \draw [fill=blue, opacity=0.25] (5.5, 1) rectangle (6, 1.6);
\end{tikzpicture}
\pause
Encontramos que la sucesión \textcolor{red}{va aumentando en una unidad}.
\end{frame}
\begin{frame}
\frametitle{Valor encontrado}
De esta manera podemos concluir que el valor del primer número es:
\begin{align*}
(1, \, 9), \, (2, \, 8), \, (3, \, 7), \, (4 \, , \, \_\_)
\end{align*}
\begin{tikzpicture}[overlay]
  \draw [fill=blue, opacity=0.25] (2.6, 1) rectangle (3, 1.6);
  \draw [fill=blue, opacity=0.25] (4.1, 1) rectangle (4.5, 1.6);
  \draw [fill=blue, opacity=0.25] (5.5, 1) rectangle (6, 1.6);
  \draw [fill=blue, opacity=0.25] (7, 1) rectangle (7.5, 1.6);
\end{tikzpicture}
\pause
Queda pendiente encontrar el segundo valor de la pareja de números.
\end{frame}
\begin{frame}
\frametitle{Segundo valor de la pareja de números}
El segundo valor de la pareja de números:
\begin{align*}
(1, \, 9), \, (2, \, 8), \, (3, \, 7), \, (\_\_ \, , \, \_\_)
\end{align*}
\begin{tikzpicture}[overlay]
  \draw [fill=blue, opacity=0.25] (3.1, 1) rectangle (3.5, 1.6);
  \draw [fill=blue, opacity=0.25] (4.6, 1) rectangle (5.1, 1.6);
  \draw [fill=blue, opacity=0.25] (6, 1) rectangle (6.6, 1.6);
\end{tikzpicture}
\pause
Encontramos que la sucesión \color{red}{va disminuyendo en una unidad}.
\end{frame}
\begin{frame}
\frametitle{Valor encontrado}
Concluimos que el valor del segundo número es:
\begin{align*}
(1, \, 9), \, (2, \, 8), \, (3, \, 7), \, (4 \, , \, 6)
\end{align*}
\begin{tikzpicture}[overlay]
  \draw [fill=blue, opacity=0.25] (3.3, 1) rectangle (3.7, 1.6);
  \draw [fill=blue, opacity=0.25] (4.6, 1) rectangle (5.1, 1.6);
  \draw [fill=blue, opacity=0.25] (6, 1) rectangle (6.6, 1.6);
  \draw [fill=blue, opacity=0.25] (7.6, 1) rectangle (8.1, 1.6);
\end{tikzpicture}
\pause
Entonces ya tenemos la pareja de valores en la sucesión: $(4, 6)$.
\end{frame}
\begin{frame}
\frametitle{Solución al ejercicio}
El inciso correcto es:
\begin{align*}
1, \, 9, \, 2, \, 8, \, 3, \, 7, \, \_\_ \, , \, \_\_
\end{align*}
\begin{choices}
\choice $8, \, 2$ \\
\choice $2, \, 9$ \\
\choice $8, \, 10$ \\
\choice $6, \, 4$ \\
\choice $4, \, 6$ \\
\end{choices}
\begin{tikzpicture}[overlay]
  \node [text = cadmiumgreen] at (1, 1) {\cmark};
\end{tikzpicture}
\end{frame}

\subsection{Ejercicio 3}

\begin{frame}
\frametitle{Ejercicio 3}
Ahora resuelve: De las parejas de números propuestas, elige la que corresponda en la sucesión:
\begin{align*}
30, \, 24, \, 19, \, 15, \, 12, \, \_\_ , \, \_\_
\end{align*}
\begin{choices}
\choice $10, \, 9$ \\
\choice $10, \, 8$ \\
\choice $10, \, 7$ \\
\choice $9, \, 8$ \\
\choice $9, \, 7$ \\
\end{choices} 
\end{frame}
\begin{frame}
\frametitle{Características de la sucesión}
\begin{align*}
30, \, 24, \, 19, \, 15, \, 12, \, \_\_ , \, \_\_
\end{align*}
\setbeamercolor{item projected}{bg=blue!70!black,fg=yellow}
\setbeamertemplate{enumerate items}[circle]
\begin{enumerate}[<+->]
\item Tenemos números enteros.
\item Hay una sucesión decreciente en cada elemento.
\end{enumerate}
\pause
Una vez que revisamos las características de la sucesión, hay que ocupar la artimética para determinar el patrón de la sucesión.
\end{frame}
\begin{frame}
\frametitle{Aritmética en los elementos}
Considerando como punto de partida el primer elemento de la sucesión: $30$, el siguiente elemento \emph{es menor}: $6$.
\\
\bigskip
\pause
Veamos cuántas unidades es menor el segundo elemento.
\end{frame}
\begin{frame}
\frametitle{Aritmética en los elementos}
Diferencia entre el primer y segundo elemento de la sucesión
\begin{align*}
30 - 24 = 6
\end{align*}
\pause
Tomamos el segundo elemento y le restamos el tercero:
\begin{align*}
24 - 19 = 5
\end{align*}
\pause
Seguimos así con los demás elementos de la sucesión.  
\end{frame}
\begin{frame}
\frametitle{Aritmética en los elementos}
\begin{eqnarray*}
30 - 24 &=& 6 \\[0.5em] 
24 - 19 &=& 5 \\[0.5em] \pause
19 - 15 &=& 4 \\[0.5em] \pause
15 - 12 &=& 3
\end{eqnarray*}
\begin{tikzpicture}[overlay]
  \draw [fill=gray, opacity=0.25](6.5, 1) rectangle (7, 5);
\end{tikzpicture}
\end{frame}
\begin{frame}
\frametitle{Secuencia descendente}
Hemos identificado una secuencia descendente de una unidad en los términos.
\\
\bigskip
\pause
Por lo que hay que utilizar esa secuencia para calcular los valores pendientes.
\end{frame}
\begin{frame}
\frametitle{Completando la secuncia}
Tenemos entonces que:
\pause
\begin{eqnarray*}
12 - 2 &=& 10 \\[0.5em] \pause
10 - 1 &=& 9
\end{eqnarray*}
\pause
Entonces los siguientes elementos de la sucesión son: $10, 9$
\end{frame}
\begin{frame}
\frametitle{Respuesta correcta}
Entonces la respuesta correcta es $A)$
\begin{align*}
30, \, 24, \, 19, \, 15, \, 12, \, 10 , \, 9
\end{align*}
\begin{choices}
\choice $10, \, 9$ \\
\choice $10, \, 8$ \\
\choice $10, \, 7$ \\
\choice $9, \, 8$ \\
\choice $9, \, 7$ \\
\end{choices}
\begin{tikzpicture}[overlay]
  \node [text = cadmiumgreen] at (0.9, 4.3) {\cmark};
\end{tikzpicture}
\end{frame}

\subsection{Ejercicio 4}

\begin{frame}
\frametitle{Ejercicio 4}
De las parejas de números propuestas, elige la que corresponda en la sucesión:
\begin{align*}
\dfrac{1}{22}, \, \dfrac{1}{18}, \, \_\_ , \, \_\_, \, \dfrac{1}{6}, \, \dfrac{1}{2}
\end{align*}
\begin{choices}
\choice $\frac{1}{4}, \, \frac{1}{12}$ \\
\choice $\frac{1}{9}, \, \frac{1}{3}$ \\
\choice $\frac{1}{12}, \, \frac{1}{10}$ \\
\choice $\frac{1}{14}, \, \frac{1}{10}$ \\
\choice $\frac{1}{10}, \, \frac{1}{8}$ \\
\end{choices} 
\end{frame}

\end{document}