\input{Preambulos/preambulo_materiales}
\title{\vspace*{-2cm}Examen Conjugación de verbos\vspace{-5ex}}
\date{\today}
\begin{document}
\maketitle

\textbf{Instrucciones:} En la siguiente tabla se presenta en la primera columna un verbo conjugado, deberás de anotar en cada una de las siguientes columnas la información que corresponda.
\begin{table}[H]
\renewcommand{\arraystretch}{1.5}
\centering
\begin{tabular}{| l | l | l | l | p{1.5cm} | p{1.5cm} |} \hline
Verbo & Persona & Número & Tiempo de conjugación & Modo & Forma \\ \hline
Había amado & & & & & \\ \hline    
Comeríamos & & & & & \\ \hline    
Perdiendo & & & & & \\ \hline    
Compraste & & & & & \\ \hline    
Viajaba & & & & & \\ \hline    
Salté & & & & & \\ \hline    
Hayan amado & & & & & \\ \hline    
Hubieran desayunado & & & & & \\ \hline    
Escribe & & & & & \\ \hline    
Beben & & & & & \\ \hline    
Cantó & & & & & \\ \hline    
Has dicho & & & & & \\ \hline    
Serán & & & & & \\ \hline    
Habrán comprado & & & & & \\ \hline    
Leas & & & & & \\ \hline    
Lleguen & & & & & \\ \hline    
Habían regalado & & & & & \\ \hline    
Anduviste & & & & & \\ \hline    
Tengan & & & & & \\ \hline    
Habrás localizado & & & & & \\ \hline    
\end{tabular}
\end{table}

\newpage

\textbf{Instrucciones: } Escoge la palabra homófona que complete correctamente cada enunciado y escríbela en la línea que corresponda:
\begin{enumerate}
\item Para que no \rule{1.75cm}{0.3mm} inundaciones, cambiaron el \rule{1.75cm}{0.3mm} del canal. \\ (cauce - cause)
\item Hoy no puedo ir a escalar a la \rule{1.75cm}{0.3mm} porque la puerta de mi auto no \rule{1.75cm}{0.3mm}. (cierra - sierra)
\item Yo cumplo siempre \rule{1.75cm}{0.3mm} mi trabajo ordenadamente, por eso \rule{1.75cm}{0.3mm} en la empresa donde presto mis servicios. (asciendo - haciendo)
\item Hay que \rule{1.75cm}{0.3mm} el pantalón antes de que la rotura se haga más grande. \\ (cocer - coser)
\item La \rule{1.75cm}{0.3mm} crece mucho en el campo cuando llueve. (hierba - hierva)
\item Es necesario que el agua \rule{1.75cm}{0.3mm} para poder \rule{1.75cm}{0.3mm} los alimentos. \\ (hierba - hierva) (cocer - coser)
\item Unas \rule{1.75cm}{0.3mm} se pararon en la carretera obstruyendo el paso de los vehículos. (reces - reses)
\item Mi madre me regaló una corbata de \rule{1.75cm}{0.3mm} (ceda - seda)
\item Cuando me inscribí en la escuela me solicitaron una fotografía \rule{1.75cm}{0.3mm}. \\ (reciente - resiente)
\item Cuando yo \rule{1.75cm}{0.3mm} al rancho montaré la yegua \rule{1.75cm}{0.3mm}. (baya - valla - vaya)
\item Después de tres horas caminando en el desierto sentíamos que el sol nos iba a \rule{1.75cm}{0.3mm}. (abrasar - abrazar)
\item El testimonio de Don Juan no me parece muy \rule{1.75cm}{0.3mm} que digamos. \\ (verás - veraz)
\item Voy a decirle la verdad de una buena \rule{1.75cm}{0.3mm} por todas. (ves - vez)
\item Aquel lobo está tan viejo que no puede \rule{1.75cm}{0.3mm} su propia comida. \\ (casar - cazar)
\item Aquel \rule{1.75cm}{0.3mm} es mucho más rápido que el león que atrapaste ayer. \\ (ciervo - siervo)
\item \rule{1.75cm}{0.3mm} mucho decirte que el precio del \rule{1.75cm}{0.3mm} de naranjas ya aumenté. \\ (ciento - siento)
\item La \rule{1.75cm}{0.3mm} de la Cámara terminó cuando aprobaron la \rule{1.75cm}{0.3mm} de un terreno  municipal a un club deportivo. (cesión - sesión)
\end{enumerate}

\newpage

\textbf{Instrucciones: } En el siguiente texto encontrás $10$ errores ortográficos: localízalos, subráyalos y escríbelos correctamente en la líneas al final del texto.
\par
El Modernismo fue un movimiento literario surgido en Ispanoamérica a  finales del siglo XIX como reacción contra el idealismo y subgetivismo romántico en la lírica y el realismo descarnado en la prosa. A pesar de ser un movimiento americano estuvo influenciado por el simbolismo y el parnasianismo franseses.
\par
Las características principales de este movimiento literario fueron la búsqueda de la innovación en el aspecto formal, el simbolismo en algunos de sus temas (el sisne y el color azul son sus símbolos principales), el cosmopolitismo, el individualismo, el ecsotismo, la variada adjetibasión, modernas metaforas, etc. Entre las innobaciones formales que introdujeron los modernistas podemos mencionar la utilización del verso alejandrino, el uso del verso endecasílabo pero con diversa acentuasion, el uso del verso monorrimo, uso de versosn de distinta métrica, la mescla de prosa y verso, etc.

\end{document}