\documentclass[14pt]{exam}
\usepackage[utf8]{inputenc}
\usepackage[T1]{fontenc}
\usepackage[spanish]{babel}
\usepackage[autostyle,spanish=mexican]{csquotes}
\usepackage{amsmath}
\usepackage{amsthm}
\usepackage{physics}
\usepackage{tikz}
\usetikzlibrary{shapes.geometric}
\usepackage{float}
\usepackage[per-mode=symbol]{siunitx}
\usepackage{gensymb}
\usepackage{multicol}
\usepackage{chemformula}
\usepackage[left=2.00cm, right=2.00cm, top=2.00cm, 
     bottom=2.00cm]{geometry}

\usepackage[fontsize=12pt]{scrextend}
\usepackage{anyfontsize}
\usepackage{pgfplots}
\pgfplotsset{compat=1.8}

\renewcommand{\questionlabel}{\thequestion)}
\decimalpoint
\sisetup{bracket-numbers = false}


\title{\vspace*{-2cm}Examen de Repaso\vspace{-5ex}}
\date{\today}

\footer{}{\thepage}{}

\begin{document}
\maketitle

\section{Matemáticas.}

\begin{questions}
    \question Calcula el perímetro y el área de los siguientes polígonos:
        \begin{parts}
            \part
            \begin{tikzpicture}
                \node[regular polygon, draw, regular polygon sides = 6, minimum size=3cm] (p) at (0,0) {};
                \draw [fill] (0, 0) circle (1pt);
                \draw (p.south) -- (p.center) node [midway, right] {$\SI{3}{\centi\meter}$};
                \node [below of=p, node distance=1.5cm] (etiqueta) {$\SI{2}{\centi\meter}$};
            \end{tikzpicture}
            Perímetro = \rule{3cm}{0.1mm} \hspace{1cm} Área = \rule{3cm}{0.1mm} \hspace{2cm}
            \part
            \begin{tikzpicture}
                \node[regular polygon, draw, regular polygon sides = 8, minimum size=3cm] (p) at (0,0) {};
                \draw [fill] (0, 0) circle (1pt);
                \draw (p.south) -- (p.center) node [midway, right] {$\SI{5}{\centi\meter}$};
                \node [below of=p, node distance=1.6cm] (etiqueta) {$\SI{4}{\centi\meter}$};
            \end{tikzpicture}
            Perímetro = \rule{3cm}{0.1mm} \hspace{1cm} Área = \rule{3cm}{0.1mm} \hspace{2cm}
        \end{parts}
    \question ¿Qué es el máximo común divisor (m.c.d.) de dos o más números?
    \vspace{2em}
    \question ¿Qué es el mínimo común múltiplo (m.c.m.) de dos o más números?
    \vspace{2em}
    \question ¿Cuál es el m.c.d. de $18$ y $24$?
    \\[0.5em]
    \begin{oneparchoices}
        \choice $6$
        \choice $3$
        \choice $12$
        \choice $4$
    \end{oneparchoices}
    \question ¿Cuál es el m.c.d. de $15$ y $30$?
    \\[0.5em]
    \begin{oneparchoices}
        \choice $3$
        \choice $5$
        \choice $15$
        \choice $30$
    \end{oneparchoices}
    \question ¿Cuál es el m.c.m. de $9$ y $6$?
    \\[0.5em]
    \begin{oneparchoices}
        \choice $12$
        \choice $15$
        \choice $18$
        \choice $54$
    \end{oneparchoices}
    \question ¿Cuál es el m.c.m. de $10$, $12$ y $15$?
    \\[0.5em]
    \begin{oneparchoices}
        \choice $60$
        \choice $80$
        \choice $100$
        \choice $120$
    \end{oneparchoices}
    \question ¿Cuál es el m.c.d. de los siguientes monomios: $a^{2} x^{2}$ y $3 a^{3} b x$? \\[0.5em]
    R = \rule{3cm}{0.1mm}
    \question ¿Cuál es el m.c.m de los siguientes monomios: $a^{2} x^{2}$ y $a^{3} x$? \\[0.5em]
    R = \rule{3cm}{0.1mm}
\end{questions}

\newpage

\section{Física.}

\begin{questions}
    \question ¿Cuáles de las siguientes son características de un vector?
    \\[0.5em]
    I. Intensidad. \\
    II. Dirección. \\
    III. Sentido. \\
    IV. Polaridad. \\
    V. Magnitud. \\[0.5em]
    \begin{oneparchoices}
        \choice I, II, y III.
        \choice II, III y IV.
        \choice II, III y V.
        \choice I, IV y V.
    \end{oneparchoices}
    \question Cuando sumamos o restamos vectores, ¿cuál es el nombre del vector que se obtiene?
    \\[0.5em]
    \begin{oneparchoices}
        \choice Complemento.
        \choice Suplemento.
        \choice Resultante.
        \choice Sobrante.
    \end{oneparchoices}
    \question Considera el siguiente vector:
    \begin{figure}[H]
        \centering
        \begin{tikzpicture}
            \draw [-stealth] (0, 0) -- (3, 0) node [above, pos=1.1] {$x$};
            \draw [-stealth] (0, 0) -- (0, 3) node [left, pos=1.1] {$y$};
            \draw [-stealth, thick] (0, 0) -- (2, 2) node [midway, above] {$a$};
            \draw (0.5, 0) arc(0:45:0.5) node [right] {$\alpha$};
        \end{tikzpicture}
    \end{figure}
    ¿Cómo quedan definidas las componente del vector $\va{a}$ en la dirección $x$ ($a_{x}$) y en dirección $y$ ($a_{y}$)?
    \begin{choices}
        \choice $a_{x} = a \, \cos \alpha$, $a_{y} = a \, \text{sen} \alpha$
        \choice $a_{x} = a \, \text{sen} \alpha$, $a_{y} = a \, \cos \alpha$
        \choice $a_{x} = a \, \text{sen} \alpha$, $a_{y} = a \, \tan \alpha$
        \choice $a_{x} = a \, \tan \alpha$, $a_{y} = a \, \cos \alpha$
    \end{choices}
    \question Considera la suma de los siguientes vectores $\va{a}$ y $\va{b}$:
    \begin{figure}[H]
        \centering
        \begin{tikzpicture}
            \draw [very thin] (0, 0) grid (3, 3);
            \draw [-stealth] (0, 0) -- (3, 0) node [above, pos=1.1] {$x$};
            \draw [-stealth] (0, 0) -- (0, 3) node [left, pos=1.1] {$y$};
            \draw [-stealth, thick] (0, 0) -- (2, 0) node [midway, below] {$\va{a}$};
            \draw [-stealth, thick] (2, 0) -- (2, 2) node [midway, right] {$\va{b}$};
        \end{tikzpicture}
        \caption{Cada cuadro en la rejilla mide $1$ unidad.}
    \end{figure}
    ¿Cuál es la magnitud $M$ y dirección $\alpha$ del vector resultante?
    \begin{choices}
        \choice $M = 2 \sqrt{2}$, \quad $\theta = \ang{90}$
        \choice $M = 8 \sqrt{2}$, \quad $\theta = \ang{225}$
        \choice $M = 2 \sqrt{2}$, \quad $\theta = \ang{225}$
        \choice $M = \dfrac{2}{\sqrt{2}}$, \quad $\theta = \ang{225}$
    \end{choices}
    \question Sabemos que la velocidad se expresa como el desplazamiento de un objeto en un tiempo determinado. 
    \begin{parts}
        \part Si la velocidad del tren bala es de $\SI{603}{\kilo\meter\per\hour}$. \\[0.5em]
        ¿Qué distancia ha recorrido en $\SI{1}{\minute}$? R = \rule{3cm}{0.1mm}
        \part El corredor jamaicano Usain Bolt estableció el récord de velocidad en la carrera de $\SI{100}{\meter}$ con un tiempo de $\SI{9.58}{\second}$. \\[0.5em]
        ¿Cuál fue su velocidad expresada en $\SI{}{\kilo\meter\per\hour}$? R = \rule{3cm}{0.1mm}
        \part La moto Ducatti Superleggera V4 corre a un máximo de $200$ millas por hora. \\[0.5em]
        A la máxima velocidad, ¿cuánto tiempo (en segundos) tarda en recorrer $\SI{350}{\meter}$? \\[0.5em]
        R = \rule{3cm}{0.1mm}
    \end{parts}
    \question ¿Cuál es el área (en metros cuadrados) de una cancha de futbol americano que mide $120$ yardas de largo por $53.33$ yardas de ancho? \\[0.5em]
    (Recuerda que $1$ yarda = $36$ pulgadas, y $1$ pulgada = $\SI{2.54}{\centi\meter}$)
    \\[0.5em]
    \begin{oneparchoices}
        \choice $\SI{5400.2}{\square\meter}$
        \choice $\SI{5363.6}{\square\meter}$
        \choice $\SI{5636.3}{\square\meter}$
        \choice Otro valor: \rule{3cm}{0.1mm}
    \end{oneparchoices}
    \question Relaciona las columnas indicando para cada propiedad, el tipo de materia a la que se relaciona: sólida, líquida o gaseosa:
    \begin{table}[H]
        \renewcommand{\arraystretch}{1.2}
        \centering
        \begin{tabular}{l | l}
            \multicolumn{1}{c}{\textbf{Propiedad}} & \multicolumn{1}{|c}{\textbf{Tipo de materia}} \\ \hline
            Compresibilidad & \\ \hline
            Ductilidad & \\ \hline
            Punto de congelación & \\ \hline
            Punto de fusión & \\ \hline
            Dureza & \\ \hline
            Licuefacción & \\ \hline
            Punto de ebullición & \\ \hline
            Presión & \\ \hline
            Maleabilidad & \\ \hline
            Compresibilidad & \\ \hline
            Condensación & \\ \hline
            Viscosidad & \\ \hline
        \end{tabular}
    \end{table}
    \newpage
    \question Completa la siguiente figura indicando los cambios de estado en la materia:
    \begin{figure}[H]
        \centering
        \begin{tikzpicture}
            \draw (0, 0) rectangle (2, 1) node [pos=0.5] {Sólido};
            \draw (4, 0) rectangle (6, 1) node [pos=0.5] {Líquido};
            \draw (8, 0) rectangle (10, 1) node [pos=0.5] {Gas};
        \end{tikzpicture}
    \end{figure}
\end{questions}


\end{document}