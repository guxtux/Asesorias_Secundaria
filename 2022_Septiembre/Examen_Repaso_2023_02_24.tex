\documentclass[14pt]{article}
\usepackage[utf8]{inputenc}
\usepackage[T1]{fontenc}
\usepackage[spanish]{babel}
\usepackage[autostyle,spanish=mexican]{csquotes}
\usepackage{amsmath}
\usepackage{amsthm}
\usepackage{physics}
\usepackage{tikz}
\usepackage{float}
\usepackage[per-mode=symbol]{siunitx}
\usepackage{gensymb}
\usepackage{multicol}
\usepackage{chemformula}
\usepackage[left=2.00cm, right=2.00cm, top=2.00cm, 
     bottom=2.00cm]{geometry}

\usepackage[fontsize=14pt]{scrextend}
\usepackage{anyfontsize}
\usepackage{pgfplots}
\pgfplotsset{compat=1.8}

% \renewcommand{\questionlabel}{\thequestion)}
\decimalpoint
\sisetup{bracket-numbers = false}


\title{\vspace*{-2cm}Examen de Repaso\vspace{-5ex}}
\date{\today}

% \footer{}{\thepage}{}

\begin{document}
\maketitle

\section{Matemáticas.}

A continuación se presentan varios sistemas de ecuaciones lineales, resuelve cada uno de ellos ya sea con el método de sustitución (sumas o restas) o con el método de reducción, posteriormente elabora una gráfica de las rectas que representan cada sistema, señalando la solución del sistema con un punto (en caso de que el sistema admita una solución. En los casos que el sistema no admita una solución, explica por qué.\begin{enumerate}
\item $\begin{cases}
2 \, x - 2 = 8 \\
-2 x + 4 y = 14
\end{cases}$
\item $\begin{cases}
x + y = 2 \\
2 \, x + 2 \, y =4
\end{cases}$
\item $\begin{cases}
7 \, x - 15 \, y = 1 \\
- x - 6 y = 8
\end{cases}$
\item $\begin{cases}
x + y = 1 \\
x + y = 3
\end{cases}$
\item $\begin{cases}
\dfrac{3 \, x}{2} + y = 11 \\[0.5em]
x + \dfrac{y}{2} = 7 
\end{cases}$
\item $\begin{cases}
3 (x + 2) = 2 \, y \\
2 (y + 5) = 7 \, x
\end{cases}$
\item Hallar el m.c.d$(28 a^{2} b^{3} c^{4}, 35 a^{3} b^{4} c^{5}, 42 a^{4} b^{5} c^{6})$
\item Hallar el m.c.d$(72 x^{3} y^{4} z^{4}, 96 x^{2} y^{2} z^{3}, 120 x^{4} y^{5} z^{7})$
\item Hallar el m.c.m$(6 a^{2}, 9 x, 12 a y^{2}, 18 x^{3} y)$
\item Hallar el m.c.m$(15 m n^{2}, 10 m^{2}, 20 n^{3}, 25 m n^{4})$
\end{enumerate}

\section{Física.}

\begin{enumerate}
\item Ordena las siguientes cinco cantidades de la más grande a las más pequeña: \break \hfill a) $\SI{0.032}{\kilo\gram}$, b) $\SI{15}{\gram}$, c) $\SI{2.7e5}{\milli\gram}$, d) $\SI{4.1e-8}{\giga\gram}$, e) $\SI{2.7e8}{\micro\gram}$. Si dos de las masas son iguales, déjalas en el mismo lugar de la lista.
\item ¿Qué distancia se desplaza hacia adelante un automóvil que se mueve a razón de $\SI{88}{\kilo\meter\per\hour}$ durante $\SI{1}{\second}$ de tiempo?
\item Un lanzador de béibsol lanzó una bola rápida a una velocidad horizontal de $\SI{160}{\kilo\meter\per\hour}$. ¿Qué tiempo le tomó a la bola llegar a la base, que está a una distancia de $\SI{18.4}{\meter}$?
\item Carl Lewis un corredor estadounidense corrió los $\SI{100}{meter}$ planos en aproximadamente $\SI{10}{\second}$, mientras que el corredor Bill Rodgers corrió el maratón ($26$ millas con $385$ yardas) en aproximadamente $\SI{2}{\hour}$ $\SI{10}{\minute}$.
\\
a) ¿Cuál es la velocidad promedio de cada uno de los corredores? \\
b) Si Carl Lewis pudiera mantener la velocidad de su carrera durante un maratón, ¿cuánto le tomaría llegar a la meta?
\item Una pieza sólida de plomo tiene una masa de $\SI{23.94}{\gram}$ y un volumen de $\SI{2.10}{\cubic\centi\meter}$. A partir de estos datos, calcula la densidad del plomo en un unidades del Sistema Internacional ($\SI{}{\kilo\gram\per\cubic\centi\meter}$)
\end{enumerate}

\end{document}