\documentclass[12pt]{article}
\usepackage[utf8]{inputenc}
\usepackage[spanish]{babel}
\usepackage{amsmath}
\usepackage{amsthm}
\usepackage{hyperref}
\usepackage{graphicx}
\usepackage{color}
\usepackage{float}
\usepackage{multicol}
\usepackage{enumerate}
\usepackage{anyfontsize}
\usepackage{anysize}
\usepackage{tikz}
\usepackage{siunitx}
\usetikzlibrary{arrows.meta, positioning}
\setlength{\parskip}{1em}
\spanishdecimal{.}
%\renewcommand{\baselinestretch}{1.5}
%\marginsize{1.5cm}{1.5cm}{1cm}{2cm}
%\author{M. en C. Gustavo Contreras Mayén. \texttt{curso.fisica.comp@gmail.com}\\
%M. en C. Abraham Lima Buendía. \texttt{abraham3081@ciencias.unam.mx}}
\title{\vspace*{-2cm}Funciones trigonométricas \vspace{-5ex}}
\date{ }
\begin{document}
\maketitle
Consideremos el triángulo rectángulo $ABC$. Las llamadas funciones o razones trigonométricas del ángulo agudo $B$ son las siguientes:
\begin{figure}[H]
\centering
\begin{tikzpicture}
	\draw (0, 0) -- (5, 0) -- (5, 4) -- cycle;
	\node at (-0.4, 0) {$B$};
	\node at (2.5, -0.4) {$c$};
	\node at (5.4, 0) {$A$};
	\node at (5.4, 2) {$b$};
	\node at (5.4, 4) {$C$};
	\node at (2.5, 2.4) {$a$};
\end{tikzpicture}
\end{figure}

\noindent
\textbf{Seno}: Es la razón entre el cateto opuesto y la hipotenusa. El seno del ángulo agudo $B$ se escribe $\text{sen} B$. Se calcula como:
\begin{align*}
\text{sen} B = \dfrac{b}{a}
\end{align*}
\textbf{Coseno}: Es la razón entre el cateto adyacente y la hipotenusa. El coseno del ángulo agudo $B$ se escribe $\cos B$. Se calcula como:
\begin{align*}
\cos B = \dfrac{c}{a}
\end{align*}
\textbf{Tangente}: Es la razón entre el cateto opuesto y cateto adyacente. Se abrevia $\tan B$. Se calcula como:
\begin{align*}
\tan B = \dfrac{b}{a}
\end{align*}
\textbf{Ejercicio 1:} Dado un triángulo rectángulo cuyos catetos miden $a = 6$ y $b = 8$ cm, calcula las funciones trigonométricas del ángulo agudo mayor.
\begin{figure}[H]
\centering
\begin{tikzpicture}
	\draw (0, 0) -- (4, 0) node [below, midway] {$a$};
	\draw (4, 0) -- (4, 5) node [right, midway] {$b$};
	\draw (4, 5) -- (0, 0) node [above=0.5, midway] {$c$};
	\node at (-0.4, 0) {$B$};
	\node at (4.4, 0) {$A$};
	\node at (4.4, 5.4) {$C$};

\end{tikzpicture}
\end{figure}
\noindent
\textbf{Solución: }
\begin{enumerate}
\item  Por medio del Teorema de Pitágoras, calculamos la hipotenusa $c$:
\begin{align*}
c = \hspace{10cm}
\end{align*}
\item El ángulo agudo mayor es: \rule{1.5cm}{0.7pt}, por que: \rule{5cm}{0.7pt}
\item 
Las funciones trigonométricas del ángulo \rule{1.5cm}{0.7pt} son:
\begin{align*}
\text{sen} \quad = \hspace{3cm} \cos \quad = \hspace{3cm} \tan \quad = 
\end{align*}
\item ¿Cuánto mide en grados el ángulo \rule{1.5cm}{0.7pt}? R: \rule{1.5cm}{0.7pt}
\item 
¿Cuánto mide en grados el otro ángulo agudo \rule{1.5cm}{0.7pt}? R: \rule{1.5cm}{0.7pt}
\end{enumerate}
\textbf{Ejercicio 2: } En la figura se muestra un vector, calcula la magnitud y la dirección del mismo (el ángulo $\alpha$).
\begin{figure}[H]
\centering
\begin{tikzpicture}
	\draw (0, 0) -- (4, 0);
	\draw (0, 0) -- (0, 5);

	\foreach \x in {1, 2, 3}
	{	\draw (\x, -0.2) -- (\x, 0.2);
		\node at (\x, -0.4) {\x};
	}

	\foreach \y in {1, 2, 3, 4}
	{	\draw (-0.2, \y) -- (0.2, \y);
		\node at (-0.4, \y) {\y};
	}

	\draw [-stealth, thick] (0, 0) -- (3, 4);
	\draw (0.5, 0) arc (0:30:1);
	\node at (0.7, 0.3) {$\alpha$};
\end{tikzpicture}
\end{figure}
\textbf{Operaciones:}
\\[1em]
Magnitud: \hspace{7cm} Dirección: 
\newpage
\textbf{Ejercicio 3: } En la figura se muestra un vector, calcula la magnitud y la dirección del mismo (el ángulo $\beta$).
\begin{figure}[H]
\centering
\begin{tikzpicture}
	\draw (0, 0) -- (3, 0);
	\draw (0, 0) -- (0, -4);

	\foreach \x in {1, 2}
	{	\draw (\x, -0.2) -- (\x, 0.2);
		\node at (\x, 0.4) {\x};
	}

	\foreach \y in {1, 2, 3}
	{	\draw (-0.2, -\y) -- (0.2, -\y);
		\node at (-0.4, -\y) {-\y};
	}

	\draw [-stealth, thick] (0, 0) -- (2, -3);
	\draw (0.3, 0) arc (0:320:0.4);
	\node at (-0.7, 0.3) {$\beta$};
\end{tikzpicture}
\end{figure}
\textbf{Operaciones:}
\\[1em]
Magnitud: \hspace{7cm} Dirección: 
\vspace{4cm}
\\[1em]
\textbf{Ejercicio 4: } Marca los siguientes puntos, en un sistema coordenado e identifica el cuadrante al que pertenecen.
\begin{table}[H]
\centering
\begin{tabular}{c c c c }
 & Cuadrante & & Cuadrante \\
$A (0, 0)$ & \rule{1cm}{0.6pt} & $K (6, 0)$ & \rule{1cm}{0.6pt}  \\
$B (4, 0)$ & \rule{1cm}{0.6pt} & $L (-4, -3)$ & \rule{1cm}{0.6pt} \\
$C (3, 2)$ & \rule{1cm}{0.6pt} & $M (-3, -3)$ & \rule{1cm}{0.6pt} \\
$D (7, 2)$ & \rule{1cm}{0.6pt} & $N (-1, -3)$ & \rule{1cm}{0.6pt} \\
$E (6, 8)$ & \rule{1cm}{0.6pt} & $O (0, -3)$ & \rule{1cm}{0.6pt} \\
$F (7, 6)$ & \rule{1cm}{0.6pt} & $P (-7, -5)$ & \rule{1cm}{0.6pt} \\
$G (0, 5)$ & \rule{1cm}{0.6pt} & $Q (2, -2)$ & \rule{1cm}{0.6pt} \\
$H (-3, -3)$ & \rule{1cm}{0.6pt} & $R (2, -4)$ & \rule{1cm}{0.6pt} \\
$I (-3, 1)$ & \rule{1cm}{0.6pt} & $S (5, -4)$ & \rule{1cm}{0.6pt} \\
$J (-5, 3)$ & \rule{1cm}{0.6pt} & $T (8, -2)$ & \rule{1cm}{0.6pt} \\
\end{tabular}
\end{table}

\newpage
\textbf{Ejercicio 5: } Se tienen cuatro vectores:
\begin{enumerate}
\item \textbf{V1} = 3 unidades de magnitud, $\theta = \ang{0}$
\item \textbf{V2} = 4 unidades de magnitud, $\theta = \ang{30}$
\item \textbf{V3} = 2 unidades de magnitud, $\theta = \ang{45}$
\item \textbf{V4} = 3 unidades de magnitud, $\theta = \ang{90}$
\end{enumerate}
\textbf{¿Cuál es la magnitud y dirección del vector resultante?}. Resuelve el ejercicio con los métodos:
\begin{enumerate}
\item Gráfico.
\item Analítico.
\item Por descomposición de componentes.
\end{enumerate}
\begin{figure}[H]
	\centering
	\begin{tikzpicture}
		\draw[style=help lines, thin] (0,0) grid[step=1cm] (9,7);
	\end{tikzpicture}
\end{figure}
\end{document}