\documentclass[14pt]{exam}
\usepackage[utf8]{inputenc}
\usepackage[T1]{fontenc}
\usepackage[spanish]{babel}
\usepackage[autostyle,spanish=mexican]{csquotes}
\usepackage{amsmath}
\usepackage{amsthm}
\usepackage{physics}
\usepackage{tikz}
\usepackage{float}
\usepackage[per-mode=symbol]{siunitx}
\usepackage{gensymb}
\usepackage{multicol}
\usepackage{chemformula}
\usepackage[left=2.00cm, right=2.00cm, top=2.00cm, 
     bottom=2.00cm]{geometry}

\usepackage[fontsize=12pt]{scrextend}
\usepackage{anyfontsize}
\usepackage{pgfplots}
\pgfplotsset{compat=1.8}

\renewcommand{\questionlabel}{\thequestion)}
\decimalpoint
\sisetup{bracket-numbers = false}


\title{\vspace*{-2cm}Examen de Repaso\vspace{-5ex}}
\date{\today}

\footer{}{\thepage}{}

\begin{document}
\maketitle

\section{Matemáticas.}

\begin{enumerate}
\item ¿Cuál es el m.c.d. de $60$ $100$ y $120$? R= \rule{2cm}{0.1mm}
\item ¿Cuál es el m.c.d. de $24$, $36$ y $72$? R= \rule{2cm}{0.1mm}
\item ¿Cuál es el m.c.d. de los siguientes monomios: $36 a^{2} b^{4}$, $48 a^{3} b^{3} c$ y $60 a^{4} b^{3} m$? \\[0.5em]
R = \rule{3cm}{0.1mm}
\item ¿Cuál es el m.c.m de los siguientes monomios: $8 a b^{2} c$ y $12 a^{3} b^{2}$? \\[0.5em]
R = \rule{3cm}{0.1mm}
\item ¿Se podrán dividir tres varillas de $\SI{20}{\centi\meter}$, $\SI{24}{\centi\meter}$ y $\SI{30}{\centi\meter}$ en pedazos de $\SI{4}{\centi\meter}$ de longitud, sin que sobre ni falta nada entra cada varilla?
\item Se tienen tres cajas que contienen $1600$ libras, $2000$ libras y $3392$ libras de jabón, respectivamente. El jabón de cada caja está dividido en bloques del mismo peso y el mayor posible. ¿Cuánto pesa cada bloque y cuántos bloques hay en cada caja? \\[0.5em]
Peso del bloque: \rule{1.5cm}{0.1mm} \hspace{0.25cm} Bloques caja 1: \rule{1.5cm}{0.1mm} \hspace{0.25cm} Bloques caja 2: \rule{1.5cm}{0.1mm} \\[0.5em]
Bloques caja 3: \rule{1.5cm}{0.1mm}
\item Se tienen tres extensiones de $\SI{3675}{\square\meter}$, $\SI{1575}{\square\meter}$ y $\SI{2275}{\square\meter}$, y se quieren dividir en parcelas iguales. ¿Cuál ha de ser la superficie de cada parcela para que el número de parcelas de cada una sea el menor posible? \\[0.5em]
R= \rule{1.5cm}{0.1mm}
\item ¿Cuál es la menor suma de dinero que se puede tener en billetes de $\$20$, $\$50$ y $\$200$ y cuántos billetes de cada denominación harían falta en cada caso?
\item Halla el menor número de bombones necesario para repartir en tres salones de $20$ alumnos, $25$ alumnos y $30$ alumnos, de modo que a cada alumno reciba el número exacto de bombones y cuántos bombones recibirían los alumnos de cada salón. \\[0.5em]
Bombones: \rule{1.5cm}{0.1mm} \hspace{1cm} Salón 1: \rule{1.5cm}{0.1mm} \hspace{1cm} Salón 2:\rule{1.5cm}{0.1mm} \\[0.5em]
Salón 3: \rule{1.5cm}{0.1mm}
\item Halla el m.c.d. de $6 a^{2} b^{3}$, $15 a^{3} b^{4}$ R = \rule{2cm}{0.1mm}
\item Halla el m.c.d. de $15 a^{2} b^{3} c$, $24 a b^{2}x$, $36 b^{4} x^{2}$ R = \rule{2cm}{0.1mm}
\item Halla el m.c.m. de $9 a x^{3} y^{4}$, $15 x^{2} y^{5}$ R = \rule{2cm}{0.1mm}
\item Halla el m.c.m. de $3 x^{2} y^{3} z$, $4 x^{3} y^{3} z^{2}$, $6 x^{4}$ R = \rule{2cm}{0.1mm}
\end{enumerate}

\end{document}