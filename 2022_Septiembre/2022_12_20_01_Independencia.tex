\input{Preambulos/preambulo_materiales}
\title{\vspace*{-2cm}Independencia de México\vspace{-5ex}}
\date{\today}
\begin{document}
\maketitle

A continuación se presentan de manera resumida, varios eventos que se sucedieron durante la Independencia de México.

En un mapa identifica los estados en donde se llevaron a cabo esos eventos, es decir, aunque se indica ya sea una ciudad o poblado, en el mapa señala con un color el estado correspondiente, en caso de que la información no esté completa, apóyate complementando la ubicación con una fuente adicional: libro de historia, una búsqueda en internet, etc.

\section*{Año 1808:}

\begin{itemize}
\item 14/08 - Llega a México la noticia de la abdicación en Bayona de los reyes de España Carlos IV y Fernando VII.
\item 19/08 - La Representación del Ayuntamiento de México que postula el retorno de la soberanía al pueblo, al no haber rey.
\item 16/08 - Se asalta el palacio virreinal y se toma preso al virrey Iturrigaray.
\end{itemize}

\section*{Año 1809:}

\begin{itemize}
\item 21/12 - Se descubre la conspiración de Valladolid.
\end{itemize}

\section*{Año 1810:}

\begin{itemize}
\item 16/09 - Levantamiento armado en Dolores dirigido por Miguel Hidalgo quien enarbola el estandarte de la Virgen de Guadalupe como enseña del movimiento.
\item 28/08 - Los insurgentes asaltan la alhóndiga de Granaditas muriendo el intendente Juan Antonio Riaño.
\item 20/10 - José María Morelos es nombrado Lugarteniente y recibe el encargo de levantar la costa del sur.
\item 30/10 - Victoria de los insurgentes en el Monte de las Cruces.
\item 07/11 - Los hermanos Galeana se unen a José María Morelos.
\item 26/11 - Hidalgo entra a Guadalajara.
\end{itemize}

\section*{Año 1811:}

\begin{itemize}
\item 21/03 - Hidalgo, Allende, Aldama, Jiménez y Abasolo son traicionados y hechos prisioneros por Elizondo en las Norias de Baján.
\item 26/06 - Allende, Aldama y Jiménez son fusilados en Chihuahua.
30/07 - Hidalgo es fusilado.
\end{itemize}

\section*{Año 1812:}

\begin{itemize}
\item 28/06 - Morelos recibe el título de Capitán General.
\end{itemize}

\section*{Año 1813:}

\begin{itemize}
\item 12/04 - Morelos ataca y toma Acapulco.
\item 14/09 - Se inaugura el Congreso de Chilpancingo y se leen los Sentimientos de la Nación escritos por Morelos.
\end{itemize}

\section*{Año 1814:}

\begin{itemize}
\item 22/10 - Se promulga en Apatzingán el decreto constitucional para la libertad de la América mexicana.
\end{itemize}

\section*{Año 1815:}

\begin{itemize}
\item 22/12 - Morelos es fusilado en San Cristóbal Ecatepec.
\end{itemize}

\section*{Año 1817:}

\begin{itemize}
\item 22/04 - Xavier Mina desembarca en Soto la Marina y lanza un manifiesto contra la tiranía de Fernando VII.
\end{itemize}

\section*{Año 1820:}

\begin{itemize}
\item 31/12 - Los habitantes del pueblo de San Diego, cerca de Veracruz, se sublevan siguiendo la proclama de Guadalupe Victoria.
\end{itemize}

\section*{Año 1821:}

\begin{itemize}
\item 24/02 - Iturbide proclama el plan de Iguala y organiza el ejército Trigarante que defenderá, las tres garantías: religión, independencia y unión.
\item 19/08 - Enfrentamiento en Atzcapotzalco, es la última acción de armas en la guerra de independencia.
\item 24/08 - Se firman los tratados de Córdoba entre don Juan de O’Donojú y Agustín de Iturbide en donde se reconoce la independencia de México.
\item 27/09 - Entrada del Ejército Trigarante a la ciudad de México. Se consuma la independencia de México.
\end{itemize}

\end{document}