\documentclass[14pt]{extarticle}
\usepackage[utf8]{inputenc}
\usepackage[T1]{fontenc}
\usepackage[spanish]{babel}
\usepackage{amsmath}
\usepackage{amsthm}
\usepackage{physics}
\usepackage{tikz}
\usepackage{float}
\usepackage[per-mode=symbol]{siunitx}
\usepackage{gensymb}
\usepackage{multicol}
\usepackage[left=2.00cm, right=2.00cm, top=2.00cm, 
     bottom=2.00cm]{geometry}

%\renewcommand{\questionlabel}{\thequestion)}
\decimalpoint
\sisetup{bracket-numbers = false}
\begin{document}
\begin{center}
\today
\end{center}

Resolver cada una de las siguientes operaciones algebraicas, en el cuardeno u hoja donde se haga el desarrollo, debe de realizarse cada paso, la solución en la misma hoja debe de estar enmarcada para identificarla.

\section{Multiplicación.}

\begin{enumerate}
\item $(m^{3} - 4 m + m^{2})(x^{2} - 2 x + 3) =$ 
\item $(a^{3} +5 a+ 2)(a^{2} - a + 5) = $
\item $(x^{n+1} - 2 x^{n+2} - x^{n+3})(x^{2} + x) = -x^{n+5} - 3 x^{n+4} - x^{n+3} + x^{n+2}$
\item $(a^{n} + b^{n})(a^{m} + b^{m}) = $
\item $(a^{n-1} - b^{n-1})(a - b) = $
\item $\left( \dfrac{2}{5} m^{2} + \dfrac{1}{3} m n - \dfrac{1}{2} n^{2} \right) \left( \dfrac{3}{2} m^{2} + 2 n^{2} - m n \right) = $
\item $\left( \dfrac{3}{8} x^{2} + \dfrac{1}{4} x - \dfrac{2}{5} \right) \left( 2 x^{3} - \dfrac{1}{3} x + 2 \right) = $
\item $\left( \dfrac{3}{4} m^{3} - \dfrac{1}{2} m^{2} n + \dfrac{2}{5} m n^{2} - \dfrac{1}{4} n^{3} \right) \left( \dfrac{2}{3} m^{3} + \dfrac{5}{2} n^{2} - \dfrac{2}{3}\right) = $
\end{enumerate}

\section{Simplifica.}

\begin{enumerate}
\item $4 (a + 5)(a - 3) = $
\item $3 a^{2} (x + 1)(x - 1) =$
\item $2 (a - 3)(a - 1)(a + 4) =$
\item $m (m - 4)(m - 6)(3 m + 2) =$
\item $a (a - 1)(a - 2)(a - 3) = $
\item $(x^{2} - 3)(x^{2} + 2 x + 1)(x - 1)(x^{2} + 3) =$
\item $4 (x + 3) +  5 (x + 2) =$
\item $a (a - x) +  3 a (x^{2} + 2 a) - a (x - 3 a) =$
\item $(m + n)^{2} - (2 m + n)^{2} + (m - 4 n)^{2} =$
\item $(a + b - c)^{2} + (a - b + c)^{2} + (a + b + c)^{2} =$
\end{enumerate}

\section{División.}

\begin{enumerate}
\item $\left( \dfrac{1}{2} x^{2} \right) \divisionsymbol \left( \dfrac{2}{3} \right) =$
\item $\left( - \dfrac{3}{5} a^{3} b \right) \divisionsymbol \left( - \dfrac{4}{5} a^{2} b \right) =$
\item $\left( - \dfrac{3}{8} c^{3} d^{5} \right) \divisionsymbol \left( \dfrac{3}{4} d^{4} \right) =$
\item $\left( \dfrac{1}{2} x^{2} - \dfrac{2}{3} x \right) \divisionsymbol \left( \dfrac{2}{3} x \right) =$
\item $\left( \dfrac{1}{3} a^{3} - \dfrac{3}{5} a^{2} + \dfrac{1}{4} a \right) \divisionsymbol \left( - \dfrac{3}{5} \right) =$
\item $\left( \dfrac{2}{3} x^{4} y^{3} - \dfrac{1}{5} x^{3} y^{4} + \dfrac{1}{4} x^{2} y^{6} \right) \divisionsymbol \left( - \dfrac{1}{5} x y^{3} \right) =$
\item $\left( \dfrac{1}{6} a^{2} + \dfrac{5}{36} a b - \dfrac{1}{6} b^{2} \right) \divisionsymbol \left( \dfrac{1}{3} a + \dfrac{1}{2} b \right) =$
\end{enumerate}

\section{Halla el valor numérico.}

Calcula la expresión, si $a = -1$, $b = 2$ y $c = - \dfrac{1}{2}$

\begin{enumerate}
\item $a^{2} - 2 a b + b^{2} =$
\item $3 a^{3} - 4 a^{2} b + 3 a b^{2} - b^{3} =$
\item $\dfrac{a b}{c} + \dfrac{a c}{b} - \dfrac{b c}{a} = \dfrac{13}{4} =$
\item $(a - b)^{2} + (b - c)^{2} - (a - c)^{2} =$
\end{enumerate}

Calcula la expresión si, $a = 2$, $b = \dfrac{1}{3}$, $x = -2$, $y = -1$, $m = 3$, $n = \dfrac{1}{2}$

\begin{enumerate}
\item $\dfrac{x^{4}}{8} - \dfrac{x^{2} y}{2} + \dfrac{2 x y^{2}}{2} - y^{3} =$
\item $(a - x)^{2} + (x - y)^{2} + (x^{2} - y^{2}) (m + x - n) = \dfrac{37}{2} =$
\item $(3 x - 2 y) + (2 n - 4 m) + 4 x^{2} y^{2} - \dfrac{x - y}{2} = \dfrac{3}{2} =$
\item $\dfrac{4 x}{3 y} - \dfrac{x^{3}}{2 + y^{3}} + \left( \dfrac{1}{n} - \dfrac{1}{b} \right) x + x^{4} - m = \dfrac{77}{3} =$
\item $\dfrac{3 a}{x} + \dfrac{2 y}{m} + \dfrac{3 n}{y} - \dfrac{m}{n} + 2 (x^{3} - y^{2} + 4) =$
\end{enumerate}

\section{Obtén el valor de $x$.}

\begin{enumerate}
\item $\dfrac{3 x}{2} + \dfrac{1}{4} =$
\item $3 (x - 7) + 5 = 2 x + 4 \hspace{0.5cm} x =$
\item $\dfrac{3 x + 2}{2} = 7 \hspace{0.5cm} x =$
\item $10 x + 2 x + 3 (x + 8) - 30 = 0 \hspace{0.5cm} x =$
\item $5 x - 55 = 2 (18 - x) \hspace{0.5cm} x =$
\item $\sqrt{2 x + 1} = 5 \hspace{0.5cm} x =2$
\item $(x + 3)^{2} + 7 = (x - 6)(x - 4) \hspace{0.5cm} x =$
\item $\dfrac{x}{2} + \dfrac{x - 4}{3} = \dfrac{3 x}{4} - 1 \hspace{0.5cm} x =$
\item $\dfrac{6}{1 -x^{2}} = \dfrac{5}{1 + x} - \dfrac{3}{1 - x} \hspace{0.5cm} x =$
\item $\dfrac{2}{5} (x - 1) + 2 = \dfrac{1}{3} (x - 2) - 6 \hspace{0.5cm} x =$
\end{enumerate}

\section{Simplifica las expresiones.}

\begin{enumerate}
\item $\dfrac{a}{2} + \dfrac{a}{4} + \dfrac{a}{8} =$
\item $\dfrac{2 y}{3} + \dfrac{y}{3} + y =$
\item $\left( \dfrac{3 x}{5} + \dfrac{8 x}{5} \right) \dfrac{15}{x} =$
\item $6 x + \dfrac{3 x}{2} - \dfrac{4 x}{3} =$
\item $\left( 2 x + \dfrac{3 x}{2} \right) \left( \dfrac{24}{x} \right) + 5 =$
\end{enumerate}

\section{Responde las siguientes preguntas.}

\begin{enumerate}
\item Si $8$ es la raíz cúbica de un número, ¿cuál es ese número?
\item Si $31$ es la raíz cuadrada de un número, ¿cuál es ese número?
\item ¿Cuál es el número cuya raíz cuarta es $4$?
\item ¿Cuál es el número cuya raíz sexta es $2$?
\item Realiza la descomposición en factores primos de:
\begin{enumerate}
\item $441$.
\item $507$.
\item $529$.
\item $1188$.
\item $2401$.
\item $2093$.
\item $2890$.
\item $15700$.
\item $20677$
\end{enumerate}
\end{enumerate}

\end{document}