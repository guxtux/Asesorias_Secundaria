\documentclass[14pt]{extarticle}
\usepackage[utf8]{inputenc}
\usepackage[T1]{fontenc}
\usepackage[spanish]{babel}
\usepackage[autostyle,spanish=mexican]{csquotes}
\usepackage{amsmath}
\usepackage{amsthm}
\usepackage{enumitem}
\usepackage{physics}
\usepackage{tikz}
\usepackage{float}
\usepackage[per-mode=symbol]{siunitx}
\usepackage{gensymb}
\usepackage{multicol}
\usepackage[left=2.00cm, right=2.00cm, top=2.00cm, 
     bottom=2.00cm]{geometry}

\decimalpoint
\sisetup{bracket-numbers = false}
\newcommand{\marcatexto}[1]{\textbf{\textit{#1}}}

\title{\vspace*{-2cm}Tiempos verbales \vspace{-5ex}}

\begin{document}
\maketitle

\section{Modo indicativo}

Sirve para expresar acciones que suceden en la realidad. Como su nombre lo dice, indica el verbo de manera literal. También podríamos decir que describe el mundo real.

\subsection{Presente (amo, como, subo)}

\begin{enumerate}[label=\alph*)]
\item  Indica que la acción significada por el verbo sucede al mismo tiempo en que uno habla:
\begin{itemize}
\item \enquote{¡Qué frío \marcatexto{hace}!}
\item \enquote{\marcatexto{Leo} este libro}
\end{itemize}
\item Significa que la acción es algo que se acostumbra hacer o es habitual:
\begin{itemize}
\item \enquote{\marcatexto{Comemos} a las dos de la tarde}
\item \enquote{Los muchachos \marcatexto{juegan} todos los domingos}
\end{itemize}
\item Manifiesta una acción que es o se considera verdadera, que pasa siempre o a la que no se le supone un límite:
\begin{itemize}
\item \enquote{La Tierra \marcatexto{gira} alrededor del Sol}
\item \enquote{El que la \marcatexto{hace} la paga}
\item \enquote{Todos los hombres \marcatexto{son} mortales}
\item \enquote{El universo \marcatexto{se} expande}
\end{itemize}
\item Hace que el tiempo de la acción se entienda como actual o próximo, o que la acción se entienda como segura:
\begin{itemize}
\item \enquote{Mis tíos \marcatexto{vienen} de Guadalajara para la Navidad}
\item \enquote{\marcatexto{Firmo} el contrato cuando te vea}
\item \enquote{Luego te lo \marcatexto{doy}}
\end{itemize}
\item Vuelve actual, para los fines del relato, una acción pasada o histórica:
\begin{itemize}
\item \enquote{Colón \marcatexto{descubre} América en 1492}
\item \enquote{Cárdenas \marcatexto{expropia} el petróleo}
\end{itemize}
\item Se usa en el antecedente (prótasis) y en el consecuente (apódosis) de las oraciones condicionales:
\begin{itemize}
\item \enquote{Si \marcatexto{estudias}, te doy un premio}
\item \enquote{Si \marcatexto{corres}, lo alcanzarás}
\end{itemize}
\item Significa mandato:
\begin{itemize}
\item \enquote{¡Te \marcatexto{bañas} de inmediato!}
\item \enquote{Cuando veas salir el Sol, me \marcatexto{avisas}}
\end{itemize}
\end{enumerate}

\subsection{Pretérito (amé, comí, subí)}

Indica que la acción significada por el verbo ya pasó, ya terminó o es anterior al momento en que se habla:
\begin{itemize}
\item \enquote{\marcatexto{Nació} en Mérida}
\item \enquote{\marcatexto{Estudió} la primaria}
\item \enquote{\marcatexto{Creí} que me caía}
\item \enquote{\marcatexto{Corrió} hasta que lo detuvieron}
\end{itemize}

\subsection{Futuro (amaré, comeré, subiré)}

\begin{enumerate}[label=\alph*)]
\item Indica que la acción se realizará después del momento en que se habla:
\begin{itemize}
\item \enquote{Te \marcatexto{llamaré} por teléfono el lunes}
\item \enquote{\marcatexto{Jugaré} muy pronto}
\end{itemize}
\item En lugar de esta forma, por lo general se usa más en México el \emph{futuro perifrástico}, que se hace con el presente de indicativo de ir, la preposición a y el infinitivo del verbo:
\begin{itemize}
\item \enquote{\marcatexto{voy a amar}}
\item \enquote{\marcatexto{voy a comer}}
\item \enquote{\marcatexto{voy a subir}}
\end{itemize}
\item Expresa la posibilidad, la probabilidad o la duda acerca de algo presente:
\begin{itemize}
\item \enquote{\marcatexto{Tendrá} unos veinte años}
\item \enquote{¿\marcatexto{Será} posible que haya guerra?}
\item \enquote{¿Qué horas \marcatexto{serán}?}
\end{itemize}
\item Indica mandato:
\begin{itemize}
\item \enquote{No \marcatexto{matarás}}
\end{itemize}
\end{enumerate}

\subsection{Copretérito (amaba, comía, subía)}

\begin{enumerate}[label=\alph*)]
\item Indica que una acción pasada es de carácter duradero o sin límites precisos:
\begin{itemize}
\item \enquote{Los niños \marcatexto{jugaban} mucho}
\item \enquote{\marcatexto{Miraba} las nubes}
\end{itemize}
\item Indica que la acción es habitual, que se acostumbra o se repite varias veces:
\begin{itemize}
\item \enquote{En aquella época \marcatexto{nadaba} a diario}
\item \enquote{\marcatexto{Disparaba} a todo lo que se movía}
\item \enquote{En mi pueblo \marcatexto{dormíamos} en hamacas}
\end{itemize}
\item Expresa una acción que sucede al mismo tiempo que otra pasada:
\begin{itemize}
\item \enquote{Cuando salí a la calle, \marcatexto{llovía}}
\item \enquote{Estuve enfermo, me \marcatexto{sentía} mal}
\end{itemize}
\item Indica que una acción pasada comenzó pero no se terminó:
\begin{itemize}
\item \enquote{\marcatexto{Salía} cuando llegó mi hermano de visita}
\item \enquote{\marcatexto{Quería} ir pero no pude}
\item \enquote{\marcatexto{Leía} y me quedé dormido}
\end{itemize}
\item Indica que una acción es dudosa, posible, deseable o que sólo sucede en la fantasía:
\begin{itemize}
\item \enquote{Creía que \marcatexto{dormías}}
\item \enquote{Pensé que \marcatexto{sufrías}}
\item \enquote{\marcatexto{Podías} haberlo dicho antes}
\item \enquote{Yo \marcatexto{era} el príncipe y tú, la princesa}
\end{itemize}
\item Se puede usar en el antecedente o en el consecuente de oraciones condicionales:
\begin{itemize}
\item \enquote{Si lo \marcatexto{hacías}, me \marcatexto{enojaba} contigo}
\item \enquote{Si me escribieras, te \marcatexto{contestaba}}
\end{itemize}
\item Expresa con cortesía una acción:
\begin{itemize}
\item \enquote{\marcatexto{Quería} pedirle un favor}
\item \enquote{¿Qué \marcatexto{deseaba}?}
\end{itemize}
\end{enumerate}

\subsection{Pospretérito (amaría, comería, subiría)}

\begin{enumerate}[label=\alph*)]
\item Indica que una acción sucede después de otra que es pasada:
\begin{itemize}
\item \enquote{Dijo que lo \marcatexto{haría} más tarde}
\item \enquote{\marcatexto{Vendría} cuando terminara la limpieza}
\end{itemize}
\item Manifiesta un cálculo sobre una acción pasada o futura, o que la acción es posible:
\begin{itemize}
\item \enquote{Cuando llegué \marcatexto{serían} las diez}
\item \enquote{\marcatexto{Bastaría} con diez pesos para comprar cacahuates}
\item \enquote{\marcatexto{Caminaría} por toda la ciudad buscándote}
\end{itemize}
\item Se usa en la consecuencia de las oraciones condicionales:
\begin{itemize}
\item \enquote{Si pudiera, lo \marcatexto{haría}}
\item \enquote{Si quisiera, lo \marcatexto{ayudaría}}
\end{itemize}
\item Expresa la acción con mucha cortesía:
\begin{itemize}
\item  \enquote{\marcatexto{Querría} pedirle un favor}
\item  \enquote{¿\marcatexto{Levantaría} su pie para sacar el mío?}
\end{itemize}
\end{enumerate}

\subsection{Antepresente (he amado, he comido, he subido)}

\begin{enumerate}[label=\alph*)]
\item  Indica que una acción, comenzada en el pasado, dura hasta el presente o tiene efectos todavía: 
\begin{itemize}
\item \enquote{Este año \marcatexto{ha llovido} mucho}
\item \enquote{\marcatexto{He decidido} renunciar}
\item \enquote{\marcatexto{Ha tenido} que ver al médico todo el año}
\item \enquote{Siempre \marcatexto{he creído} en la bondad humana}
\item \enquote{La ciencia \marcatexto{ha progresado} en este siglo}
\item \enquote{Si no \marcatexto{han pagado} para el martes, los desalojan}
\end{itemize}
\item Indica que la acción sucedió inmediatamente antes del momento presente:
\begin{itemize}
\item \enquote{\marcatexto{He dicho} que te salgas}
\end{itemize}
\end{enumerate}

\subsection{Antefuturo (habré amado, habré comido, habré subido)}

\begin{enumerate}[label=\alph*)]
\item Expresa que la acción es anterior a otra acción en el futuro, pero posterior con relación al presente:
\begin{itemize}
\item \enquote{Cuando vengas por mí, ya \marcatexto{habré terminado} el trabajo}
\item \enquote{Para el sábado \marcatexto{habré salido} de vacaciones}
\end{itemize}
\item Puede expresar duda acerca de una acción pasada:
\begin{itemize}
\item \enquote{No le \marcatexto{habrás entendido} bien}
\end{itemize}
\item En ocasiones expresa sorpresa ante una acción pasada:
\begin{itemize}
\item \enquote{¡Si \marcatexto{habré sido} tonto!}
\item \enquote{¡\marcatexto{Habráse visto} qué tontería!}
\end{itemize}
\end{enumerate}

\subsection{Antecopretérito \\ (había amado, había comido, había subido)}

Significa que la acción pasada sucedió antes que otra también ya pasada:
\begin{itemize}
\item \enquote{Me dijo que \marcatexto{había comprado} un terreno}
\item \enquote{Supuse que ya lo \marcatexto{}habías visto}
\item \enquote{¿Cómo que perdiste? ¡Tú siempre `\marcatexto{habías ganado}!}
\end{itemize}

\subsection{Antepospretérito (habría amado, habría comido, habría subido)}

\begin{enumerate}[label=\alph*)]
\item Indica que la acción sucede después de otra pasada y antes de una que, para el pasado, sería futura:
\begin{itemize}
\item \enquote{Me prometió que cuando yo fuera a recoger al niño, ella ya lo \marcatexto{habría vestido}}
\end{itemize}
\item Expresa que la acción puede haber sucedido en el pasado o la suposición de que hubiera sucedido, aunque después se compruebe que no fue así:
\begin{itemize}
\item \enquote{En aquel entonces \marcatexto{habría cumplido} 20 años}
\item \enquote{Se anunció que los bombarderos enemigos \marcatexto{habrían atacado} una población de campesinos}
\end{itemize}
\item Manifiesta la opinión o la duda acerca de una acción presente o futura:
\begin{itemize}
\item \enquote{¿\marcatexto{Habría sido} necesario el ataque?}
\item \enquote{¿\marcatexto{Habríamos creído} que fueran capaces de hacerlo?}
\end{itemize}
\item Se puede usar en la consecuencia de oraciones condicionales:
\begin{itemize}
\item \enquote{Si \marcatexto{hubiera llegado}, te habría avisado.}
\end{itemize}
\end{enumerate}

\section{Modo subjuntivo.}

Los tiempos del modo subjuntivo expresan \textbf{relaciones de anterioridad, simultaneidad o posterioridad} de las acciones, con respecto al tiempo en que sucede otra acción o al tiempo en que uno habla; por eso, aunque sus nombres —presente, pretérito, futuro, antepresente, antepretérito, antefuturo— se correspondan con los del modo indicativo, deben considerarse por separado: los del indicativo se refieren al tiempo real mientras que los del subjuntivo son relativos con respecto a ese modo.

\subsection{Presente (ame, coma, suba)}

\begin{enumerate}[label=\alph*)]
\item Significa que la acción del verbo sucede al mismo tiempo que otra o después de ella:
\begin{itemize}
\item \enquote{Cuando \marcatexto{salga}, lo atrapas}
\item \enquote{Lo quiero tanto como lo \marcatexto{quieras} tú}
\item \enquote{Deseo que \marcatexto{estés} bien}
\item \enquote{No sé si \marcatexto{cante}}
\item \enquote{No creo que \marcatexto{venga}}
\item \enquote{Que nos \marcatexto{vaya} bien}
\item \enquote{Me pidieron que \marcatexto{hable} en la junta}
\end{itemize}
\item Expresa mandato:
\begin{itemize}
\item \enquote{Que me \marcatexto{dejes} en paz}
\item \enquote{\marcatexto{Sepan} todos}
\item \enquote{¡Que se \marcatexto{callen}!}
\end{itemize}
\item Manifiesta la negación del imperativo:
\begin{itemize}
\item \enquote{Ve a casa/No \marcatexto{vayas} a casa}
\end{itemize}
\end{enumerate}

\subsection{Pretérito (amara o amase, comiera o comiese, subiera o subiese)}

\begin{enumerate}[label=\alph*)]
\item Indica que la acción del verbo sucede al mismo tiempo o después de otra, ya sea pasada, presente o futura:
\begin{itemize}
\item \enquote{El maestro le pidió que se \marcatexto{presentase} al examen}
\item \enquote{Mandó que \marcatexto{podara} los árboles}
\end{itemize}
\item Manifiesta la posiblidad de que algo suceda o haya sucedido, o una opinión acerca de ello:
\begin{itemize}
\item \enquote{Si \marcatexto{agradeciera} los favores, sería mejor}
\item \enquote{No \marcatexto{debieran} haberse molestado}
\item \enquote{Quizá \marcatexto{viniera} porque necesitara algo}
\end{itemize}
\item Se usa en las oraciones condicionales:
\begin{itemize}
\item \enquote{Si \marcatexto{tuviera} parque, no estaría usted aquí}
\item \enquote{Si \marcatexto{tuviese} dinero, me compraba una casa}
\item \enquote{Si me \marcatexto{besaras}, viviría feliz}
\end{itemize}
\item Manifiesta cortésmente un deseo o una pregunta:
\begin{itemize}
\item \enquote{\marcatexto{Quisiera} hablar con usted}
\item \enquote{Si me \marcatexto{volviese} a explicar el problema, se lo agradecería.}
\end{itemize}
\end{enumerate}

\subsection{Futuro (amare, comiere, subiere)}

\begin{enumerate}[label=\alph*)]
\item Expresa que una acción venidera es sólo posible:
\begin{itemize}
\item \enquote{Quien así lo \marcatexto{hiciere}, que la nación se lo demande.}
\end{itemize}
\item No se usa en la lengua hablada; y en la escrita, solamente en ciertos escritos legales.
\item Se usa en ciertas frases hechas, como:
\begin{itemize}
\item \enquote{Sea lo que \marcatexto{fuere}}
\item \enquote{Venga quien \marcatexto{viniere}.}
\end{itemize}
\end{enumerate}

\subsection{Antepresente (haya amado, haya comido, haya subido)}

\begin{enumerate}[label=\alph*)]
\item Expresa que la acción es pasada y terminada, y además anterior a otra:
\begin{itemize}
\item \enquote{No me dijo que \marcatexto{hayan estado} en Veracruz}
\item \enquote{Cuando \marcatexto{haya terminado} la tarea, jugaré con mis amigos}
\end{itemize}
\item Manifiesta el deseo, la suposición o la probabilidad de una acción pasada y terminada:
\begin{itemize}
\item \enquote{Ojalá \marcatexto{hayamos ganado} la votación}
\item \enquote{Que \marcatexto{hayas dicho} la verdad es importante}
\end{itemize}
\end{enumerate}

\subsection{Antepretérito \\ (hubiera o hubiese amado, hubiera o hubiese comido, hubiera o hubiese subido)}

\begin{enumerate}[label=\alph*)]
\item Indica que la acción es pasada y terminada, y anterior a otra igualmente pasada:
\begin{itemize}
\item \enquote{Lo \marcatexto{hubiese anunciado} cuando dio los otros avisos}
\end{itemize}
\item Manifiesta la posibilidad o el deseo acerca de una acción pasada:
\begin{itemize}
\item \enquote{Si lo \marcatexto{hubiera sabido}, habría venido de inmediato}
\item \enquote{¡Que \marcatexto{hubiera nacido} rico!}
\item \enquote{Si \marcatexto{hubiera venido}, la habría conocido}
\item \enquote{\marcatexto{Hubiera visto} el paisaje durante mi viaje}
\end{itemize}
\end{enumerate}

\subsection{Antefuturo (hubiere amado, hubiere comido, hubiere subido)}

\begin{enumerate}[label=\alph*)]
\item Expresa la posibilidad de que una acción haya sucedido en el futuro:
\begin{itemize}
\item \enquote{Si no \marcatexto{hubiere cumplido} mis promesas el año próximo, mereceré un castigo}
\end{itemize}
\item No se usa actualmente, con excepción de algunos textos legales.
\end{enumerate}
\end{document}