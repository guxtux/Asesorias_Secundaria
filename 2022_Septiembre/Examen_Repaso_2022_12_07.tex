\documentclass[12pt]{exam}
\usepackage[utf8]{inputenc}
\usepackage[T1]{fontenc}
\usepackage[spanish]{babel}
\usepackage[autostyle,spanish=mexican]{csquotes}
\usepackage{amsmath}
\usepackage{amsthm}
\usepackage{physics}
\usepackage{tikz}
\usepackage{float}
\usepackage[per-mode=symbol]{siunitx}
\usepackage{gensymb}
\usepackage{multicol}
\usepackage[left=2.00cm, right=2.00cm, top=2.00cm, 
     bottom=2.00cm]{geometry}

\renewcommand{\questionlabel}{\thequestion)}
\decimalpoint
\sisetup{bracket-numbers = false}
\begin{document}
\begin{center}
\today
\end{center}

\section{Español}
\begin{questions}
    \question Es el texto  que nos narra la vida de una persona. En ella se pueden abordar todos los aspectos de la personalidad estudiada: acontecimientos, etapas, estudios, ambiciones, conflictos, etc. Nos referimos a:
    \\[1em]
    \begin{oneparchoices}
        \choice Biografía.
        \choice Autobiografía.
        \choice Monografía.
        \choice Opinión.
    \end{oneparchoices}

    \question Son palabras cuyo significado es similar; gracias a ellos podemos dar variedad y precisión a un texto.
    \\[1em]
    \begin{oneparchoices}
        \choice Antónimos.
        \choice Sinónimos.
        \choice Pronombres.
        \choice Nombres.
    \end{oneparchoices}

    \question Este tipo de palabras sustituyen a los nombres o sustantivos y se clasifican según su función gramatical.
    \\[1em]
    \begin{oneparchoices}
        \choice Antónimos.
        \choice Sinónimos.
        \choice Pronombres.
        \choice Nombres.
    \end{oneparchoices}

    \question Mío, tuyo, suyo, nuestro ¿se refieren al tipo de pronombres?
    \\[1em]
    \begin{oneparchoices}
        \choice Indefinidos.
        \choice Demostrativos.
        \choice Relativos.
        \choice Posesivos.
    \end{oneparchoices}

    \question ¿Cuál de las siguientes palabras es aguda?
    \\[1em]
    \begin{oneparchoices}
        \choice Coche.
        \choice Tajín.
        \choice Cántaro.
        \choice Mesa.
    \end{oneparchoices}

    \question Solo una de las siguientes palabras está escrita correctamente ¿Cuál es?
    \\[1em]
    \begin{oneparchoices}
        \choice Ágil.
        \choice Álto.
        \choice Métros.
        \choice Alégran.
    \end{oneparchoices}

    \question El golpe de voz en las palabras sobreesdrújulas recae sobre ...
    \\[1em]
    \begin{oneparchoices}
        \choice la última sílaba.
        \choice la penúltima sílaba.
        \choice la antepenúltima sílaba.
        \choice antes de la penúltima sílaba.
    \end{oneparchoices}

    \question Solo una de las siguientes palabras está bien separada silábicamente ¿Cuál es?
    \\[1em]
    \begin{oneparchoices}
        \choice ca-í-a.
        \choice a-umen-ta.
        \choice sa-e-ta.
        \choice a-i-re.
    \end{oneparchoices}

    \newpage
    
    Analiza la siguiente ficha bibliográfica y contesta lo que se pide.
    
  
    TRIGUEROS Gaisman, María y Jaime Pimentel. \enquote{El movimiento. La descripción de los cambios en la naturaleza}, En Física, Ciencias 2, Castillo, México, 2007.

    \question Opción correcta que corresponde al título del libro y la editorial
    
    \begin{choices}
        \choice María Trigueros y Jaime Pimentel. Física. Ciencias 2.
        \choice \enquote{El movimiento. La descripción de los cambios en la naturaleza}, En Física, Ciencias 2.
        \choice Física, Ciencias 2, Castillo.
        \choice Castillo, México.
    \end{choices}

    \question Al hacer un resumen, se puede buscar las ideas principales de un texto en:
    \begin{choices}
    \choice Los enunciados que responden a las preguntas ¿quién? , ¿cuándo?, ¿Cómo?
    \choice Las palabras escritas entre comillas, en letras en cursiva (itálicas) o negritas.
    \choice Los diversos gráficos que apoyan al texto, tales como los diagramas.
    \choice Los pies de página que se encuentran al final de la página o al final del texto.
    \end{choices}

    \question \enquote{Convencer a los otros mediante la exposición de sus ideas} es la definición de:
    \\[0.5em]
    \begin{oneparchoices}
    \choice Argumentar.
    \choice Persuadir.
    \choice Inducir.
    \choice Dialogar.
    \end{oneparchoices}
    \\[0.5em]

Lee lo siguiente y contesta las siguientes preguntas.
\\[0.5em]
Art.1 En los Estados Unidos Mexicanos, todo individuo \enquote{gozará} de las garantías que \enquote{otorga} esta Constitución, las cuales no \enquote{podrán} restringirse ni suspenderse, sino en los casos y con las condiciones que  ella misma establece.

Está prohibida la esclavitud en los Estados Unidos Mexicanos. Los esclavos del extranjero que \enquote{entren} al territorio nacional \enquote{alcanzarán}, por este solo hecho, su libertad  y la protección de las leyes.
\enquote{Queda} prohibida toda discriminación por origen étnico o nacional.
\\[0.25em]

                 Constitución Política de los EUM.Méx: Esfinge 2006 pág 5.

\question El artículo anterior está contenido en un documento:
\\[0.5em]
\begin{oneparchoices}
    \choice Nacional.
    \choice Local.
    \choice Regional.
    \choice Internacional.
\end{oneparchoices}

\question En el texto anterior de acuerdo a su contenido se refiere a:
\begin{choices}
    \choice Derechos, súplicas.
    \choice Obligaciones, derechos.
    \choice Peticiones, derechos.
    \choice Compromisos, atribuciones.
\end{choices}

\newpage

\question  ¿En qué tiempo y modo verbal se encuentran las palabras  resaltadas entre comillas del texto?
\begin{choices}
\choice Presente y futuro subjuntivo.
\choice Futuro y pretérito indicativo.
\choice Futuro y  presente indicativo.
\choice Presente y futuro imperativo.
\end{choices}

\question  ¿Cómo se le llama a los datos que aparecen al final del fragmento, colocados en la parte inferior derecha?
\begin{choices}
\choice Nota literaria.
\choice Ficha de autor.
\choice Nota bibliográfica.
\choice Referencia cruzada.
\end{choices}

\medskip
Analiza con atención.
\\[0.5em]
El maestro Carlos de educación física  realizó  en el pizarrón un esquema mediante llaves y de manera  jerarquizada escribió  las ideas  principales  de los materiales que ocuparía en la clase de básquetbol.

\question ¿Qué recurso utilizó el maestro para describir la actividad?
\\[0.5em]
\begin{oneparchoices}
\choice Tablas.
\choice Ilustraciones.
\choice Cuadro sinóptico.
\choice Gráfica de barras.
\end{oneparchoices}

\medskip
Observa el poema que se muestra a continuación y contesta las siguientes preguntas.

\begin{verse}
Gris y morado \\
es mi verde olivar; \\
blanca mi casa y \\
azul mi mar.
\\[0.75em]
Cuando tú vengas \\
no me vas a encontrar; \\
yo seré un pájaro \\
del verde olivar.
\\[0.75em]
Cuando tú vengas \\
no me vas a encontrar \\
seré una llamita \\
roja del hogar.
\\[0.75em]
Cuando tú vengas \\
no me vas a encontrar; \\
seré una estrella \\
encima del mar.
\end{verse}

\question ¿Número de estrofas en el poema?
\\[0.5em]
\begin{oneparchoices}
\choice 6.
\choice 16.
\choice 4.
\choice 10.
\end{oneparchoices}

\question ¿Versos de cada estrofa en el poema?
\\[0.5em]
\begin{oneparchoices}
\choice 4
\choice 16.
\choice 6.
\choice 10.
\end{oneparchoices}

\question ¿Número total de versos en el poema?
\\[0.5em]
\begin{oneparchoices}
\choice 4.
\choice 16.
\choice 6.
\choice 10.
\end{oneparchoices}

\end{questions}
\end{document}