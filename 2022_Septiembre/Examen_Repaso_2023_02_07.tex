\documentclass[12pt]{exam}
\usepackage[utf8]{inputenc}
\usepackage[T1]{fontenc}
\usepackage[spanish]{babel}
\usepackage[autostyle,spanish=mexican]{csquotes}
\usepackage{amsmath}
\usepackage{amsthm}
\usepackage{physics}
\usepackage{tikz}
\usepackage{float}
\usepackage[per-mode=symbol]{siunitx}
\usepackage{gensymb}
\usepackage{multicol}
\usepackage{chemformula}
\usepackage[left=2.00cm, right=2.00cm, top=2.00cm, 
     bottom=2.00cm]{geometry}

\usepackage[fontsize=14pt]{scrextend}
\usepackage{anyfontsize}

\renewcommand{\questionlabel}{\thequestion)}
\decimalpoint
\sisetup{bracket-numbers = false}


\title{\vspace*{-2cm}Examen de Repaso\vspace{-5ex}}
\date{\today}

\footer{}{\thepage}{}

\begin{document}
\maketitle

\section{Español.}

Indica en qué tiempo verbal están conjugadas las frases:
\begin{questions}
    \question \emph{"Dijo que lo haría más tarde"}.
    \\[0.5em]
    \begin{oneparchoices}
        \choice Futuro.
        \choice Antepresente.
        \choice Pospretérito.
        \choice Copretérito.
    \end{oneparchoices}
    \question \emph{"He decidido renunciar"}.
    \\[0.5em]
    \begin{oneparchoices}
        \choice Antepretérito
        \choice Antepresente.
        \choice Pretérito.
        \choice Antecopretérito.
    \end{oneparchoices}
    \question \emph{"Si hubiera llegado, te habría avisado"}.
    \\[0.5em]
    \begin{oneparchoices}
        \choice Antefuturo.
        \choice Antepresente.
        \choice Presente.
        \choice Antepospretérito.
    \end{oneparchoices}
    \question \emph{"Para el sábado, habré salido de vacaciones"}.
    \\[0.5em]
    \begin{oneparchoices}
        \choice Antecopretérito.
        \choice Antefuturo.
        \choice Antepospretérito.
        \choice Antepresente.
    \end{oneparchoices}
    \question \emph{"Los niños jugaban mucho"}.
    \\[0.5em]
    \begin{oneparchoices}
        \choice Copretérito.
        \choice Pospretérito.
        \choice Pretérito.
        \choice Antepretérito.
    \end{oneparchoices}
    \question ¿Cuántos tiempos de conjugación hay en el modo indicativo?
    \\[0.5em]
    \begin{oneparchoices}
        \choice $8$.
        \choice $9$.
        \choice $10$.
        \choice $11$.
    \end{oneparchoices}
    \question ¿Cuántos tiempos de conjugación hay en el modo subjuntivo?
    \\[0.5em]
    \begin{oneparchoices}
        \choice $5$.
        \choice $6$.
        \choice $7$.
        \choice $8$.
    \end{oneparchoices}
    \question Considera los siguientes verbos: \\[0.5em]
    I. Cantar \hspace{1cm} II. Pedir \hspace{1cm} III. Contar \hspace{1cm} IV. Comer
    \\[0.5em]
    ¿Qué verbo(s) es(son) regular(es):
    \\[0.5em]
    \begin{oneparchoices}
        \choice Solo I.
        \choice I y II.
        \choice I y III.
        \choice I y IV.
    \end{oneparchoices}
    \question ¿Cuál de los siguientes verbos NO es irregular?
    \\[0.5em]
    \begin{oneparchoices}
        \choice Conducir.
        \choice Subir.
        \choice Ser.
        \choice Ir.
    \end{oneparchoices}
\end{questions}

\newpage

\section{Química.}

\begin{questions}
    \question Relaciona las columnas por medio de la letra.
\\[0.5em]
\begin{tabular}{l}
A. Punto de ebullición \\
B. Punto de fusión \\
C. Viscosidad \\
D. Densidad \\
E. Estado gaseoso \\
F. Disolución \\
G. Sustancia \\
H. Estado líquido \\
\end{tabular}
\begin{tabular}{| p{2cm} | p{9cm} |} \hline
 & El material ocupa todo el volumen del recipiente que lo contiene y sus moléculas se mueven en desorden. \\ \hline
 & Es un ejemplo de mezcla homogénea. \\ \hline
 & El material toma la forma del recipiente que lo contiene y su volumen es prácticamente fijo. \\ \hline
 & Material formado por un sólo componente. \\ \hline
 & Es la relación entre masa y volumen de una sustancia. \\ \hline
 & Es la temperatura a la cual una sustancia pasa del estado líquido al gaseoso.\\ \hline
 & A esta temperatura, un sólido cambia al estado líquido sin dejar de ser la misma sustancia. \\ \hline
 & Se refiere a la resistencia a fluir que tiene una sustancia. \\ \hline
\end{tabular}
    \question Relaciona el nombre del científico con el modelo atómico que propuso:
\\[1em]
\begin{tabular}{| p{2cm} | p{10cm} | } \hline
 & Las sustancias se pueden dividir hasta partículas indivisibles y separadas llamadas átomos. \\ \hline
 & Considera al átomo como una gran esfera con carga eléctrica positiva, en la que se distribuyen los electrones como pequeños granitos al que llamó \enquote{budín con pasas}. \\ \hline
 & El centro del átomo está constituido por el núcleo donde reside su masa con carga positiva, a la que llamó protón, y una atmósfera electrónica compuesta de órbitas indeterminadas en las que se encuentran los electrones como el sistema planetario. \\ \hline
 & Los electrones se encuentran y giran en órbitas definidas y que cada una contiene una cantidad de energía, por esta razón los llamó niveles de energía. \\ \hline
\end{tabular}
\begin{tabular}{l}
A. Dalton \\
B. Thompson \\
C . Rutherford \\
D. Bohr \\ 
\end{tabular}
    \newpage
    \question Se define al \textbf{enlace químico} como la fuerza que mantiene unidos a los átomos en un compuesto, ¿los enlaces se hacen a través de ...?
    \begin{choices}
        \choice Los electrones más internos.
        \choice Los neutrones.
        \choice Los electrones de valencia.
        \choice Los protones.
    \end{choices}
    \question Completa la siguiente tabla representando la estructura de Lewus de los siguientes elementos:
    \begin{table}[H]
        \renewcommand{\arraystretch}{2}
        \centering
        \begin{tabular}{| l | l | p{5cm}|} \hline
            \multicolumn{1}{|c}{\textbf{Elemento}} & \multicolumn{1}{|c|}{\textbf{Electrones de valencia}} & \multicolumn{1}{c|}{\textbf{Estrctura de Lewis}} \\ \hline
            Nitrógeno (N) & & \\ \hline
            Magnesio (Mg) & &  \\ \hline
            Flúor (F) & &  \\ \hline
            Aluminio (Al) & &  \\ \hline
            Calcio (Ca) & & \\ \hline
            Argón (Ar) & & \\ \hline
        \end{tabular}
    \end{table}
    
    \question Completa los datos que faltan en la formación de los siguientes compuestos. Además, indica si cada átomo recibe, cede o comparte electrones al configurar el compuesto.
    \begin{table}[H]
        \centering
        \renewcommand{\arraystretch}{2}
        \begin{tabular}{l}
            \fontsize{16}{16}\selectfont
            \ch{"\chlewis{0.}{H}" + "\chlewis{180., 90:, 0:, 270:}{Cl}" ->} \rule{3cm}{0.2mm} \\ 
            \ch{"\chlewis{0.}{Na}" + "\chlewis{180., 90:, 0:, 270:}{Cl}" ->} \rule{3cm}{0.2mm} \\ 
            \ch{"\chlewis{0:}{Ba}" + "\chlewis{90:, 0:, 270:}{S}" ->} \rule{3cm}{0.2mm} \\ 
            \ch{"\chlewis{0.}{H}" + "\chlewis{90:, 0., 180., 270:}{O}" ->} \rule{3cm}{0.2mm} \\ 
        \end{tabular}
    \end{table}
\end{questions}

    

\end{document}