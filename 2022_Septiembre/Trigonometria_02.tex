\documentclass[12pt]{article}
\usepackage[utf8]{inputenc}
\usepackage[spanish]{babel}
\usepackage{amsmath}
\usepackage{amsthm}
\usepackage{hyperref}
\usepackage{graphicx}
\usepackage{color}
\usepackage{float}
\usepackage{multicol}
\usepackage{enumerate}
\usepackage{anyfontsize}
\usepackage{anysize}
\usepackage{tikz}
\usepackage{siunitx}
\usepackage{gensymb}
\usetikzlibrary{arrows.meta, positioning}
\setlength{\parskip}{1em}
\spanishdecimal{.}
%\renewcommand{\baselinestretch}{1.5}
%\marginsize{1.5cm}{1.5cm}{1cm}{2cm}
%\author{M. en C. Gustavo Contreras Mayén. \texttt{curso.fisica.comp@gmail.com}\\
%M. en C. Abraham Lima Buendía. \texttt{abraham3081@ciencias.unam.mx}}
\title{\vspace*{-2cm}Ángulos y triángulos\vspace{-5ex}}
\date{\today}
\begin{document}
\maketitle

\section{Ángulos.}

\begin{enumerate}
\item \textbf{Ángulos complementarios: } Son dos ángulos que sumados valen un ángulo recto, es decir $\ang{90}$.
\\
Calcula los complementos de los siguientes ángulos:
\begin{enumerate}
\item $\ang{18}$ \hspace{1cm} \rule{1.5cm}{0.7pt}
\item $\ang{36;52;}$ \hspace{1cm} \rule{1.5cm}{0.7pt}
\item $\ang{48;39;15}$ \hspace{1cm} \rule{1.5cm}{0.7pt}
\end{enumerate}
\item \textbf{Ángulos suplementarios: } Son los ángulos que sumados valen dos ángulos rectos, es decir $\ang{180}$.
\item \textbf{Suplemento de un ángulo: } Es lo que le falta al ángulo para valer dos ángulos rectos.
\\
Calcula los suplementos de los siguientes ángulos:
\begin{enumerate}
\item $\ang{78}$ \hspace{1cm} \rule{1.5cm}{0.7pt}
\item $\ang{92;15;}$ \hspace{1cm} \rule{1.5cm}{0.7pt}
\item $\ang{123;9;16}$ \hspace{1cm} \rule{1.5cm}{0.7pt}
\end{enumerate}
\item Considera la siguiente figura:
\begin{figure}[H]
\centering
\begin{tikzpicture}
    \draw (0, 0) -- (5, 0) node [near end, pos=1.1] {$A$};
    \draw (0, 0) -- (0, 5) node [near end, pos=1.1] {$D$};
    \draw [rotate=20] (0, 0) -- (5, 0);
    \draw [rotate=50] (0, 0) -- (5, 0);

    \draw (1, 0) arc(0:20:1);
    \draw (1.4, 0.5) arc(0:50:1);
    \draw (0.8, 1) arc(40:90:1);

    \node at (-0.3, -0.3) {$0$};
    \node at (5, 1.7) {$B$};
    \node at (3.5, 4) {$C$};
\end{tikzpicture}
\end{figure}
Si el $\angle \, AOD$ es recto y se cumple que:
\begin{align*}
\angle \, AOB &= 2 \, x \\
\angle \, BOC &= 3 \, x \\
\angle \, COD &= 4 \, x
\end{align*}
¿Cuánto vale cada ángulo?
\item Considera la siguiente figura:
\begin{figure}[H]
\centering
\begin{tikzpicture}
    \draw [rotate=30] (-1, 0) -- (3, 0);
    \draw [rotate=30] (2, -2) -- (0, 2);

    % \draw (1, 0) arc(0:20:1);
    % \draw (1.4, 0.5) arc(0:50:1);
    % \draw (0.8, 1) arc(40:90:1);

    \node at (3, 1.5) {$A$};
    \node at (3, -1) {$B$};
    \node at (-1, -1) {$C$};
    \node at (-1.3, 1.5) {$D$};
    \node at (0.9, 0.1) {$O$};

    \node at (0.8, -0.5) {$2 \, x$};
    \node at (1.8, 0.5) {$x$};
\end{tikzpicture}
\end{figure}
Si $\angle \, BOC = 2 \, \angle \, BOA$, calcula los ángulos:
\begin{enumerate}
\item $\angle \, AOB$ \hspace{1cm} \rule{1.5cm}{0.7pt}
\item $\angle \, BOC$ \hspace{1cm} \rule{1.5cm}{0.7pt}
\item $\angle \, COD$ \hspace{1cm} \rule{1.5cm}{0.7pt}
\item $\angle \, AOD$ \hspace{1cm} \rule{1.5cm}{0.7pt}
\end{enumerate}
\item Halla el ángulo que es igual a la mitad de su suplemento.
\\ \textbf{R.} \rule{4cm}{0.7pt}
\item Halla el ángulo que es igual al doble de su suplmento.
\\ \textbf{R.} \rule{4cm}{0.7pt}
\end{enumerate}

\section{Triángulos.}

\begin{enumerate}
\item Del teorema de Pitágoras conocemos la expresión para calcular el valor de la hipotenusa $h$ en términos de los catetos $a$ y $b$:
\begin{align*}
h = \sqrt[2]{a^{2} + b^{2}}
\end{align*}
\begin{enumerate}
\item De la expresión de Pitágoras, obtén una expresión para calcular el valor de $a$ en términos de $h$ y de $b$.
\\[1em]
\textbf{R.} \rule{6cm}{0.7pt}
\item De la expresión de Pitágoras, obtén una expresión para calcular el valor de $b$ en términos de $h$ y de $a$.
\\[1em]
\textbf{R.} \rule{6cm}{0.7pt}
\end{enumerate}
\item Si $a$ es la hipotenusa y $b$, $c$ los catetos de un triángulo rectángulo, calcula el lado que falta:
\begin{enumerate}
\item $b = \SI{10}{\centi\meter}$, $c = \SI{6}{\centi\meter}$ \hspace{1cm} \rule{2cm}{0.7pt}
\item $b = \SI{30}{\centi\meter}$, $c = \SI{40}{\centi\meter}$ \hspace{1cm} \rule{2cm}{0.7pt}
\item $a = \SI{32}{\meter}$, $c = \SI{12}{\meter}$ \hspace{1cm} \rule{2cm}{0.7pt}
\item $a = \SI{32}{\meter}$, $c = \SI{20}{\meter}$ \hspace{1cm} \rule{2cm}{0.7pt}
\item $a = \SI{80}{\kilo\meter}$, $b = \SI{80}{\kilo\meter}$ \hspace{1cm} \rule{2cm}{0.7pt}
\end{enumerate}
\item En el triángulo rectángulo $\triangle \, ABC$ (donde $\angle \, A = \ang{90})$. Calcula las funciones trigonométricas de los ángulos $B$ y $C$, si $b = \SI{2}{\centi\meter}$ y $c = \SI{4}{\centi\meter}$.
\item Dados los puntos $A \, (2, 3)$ y $B \, (-1, 4)$, calcula las funciones trigonométricas de \break \hfill $\angle \, XOA$ y $\angle \, XOB$.
\begin{figure}[H]
	\centering
	\begin{tikzpicture}
		\draw[style=help lines, thin] (0, 0) grid[step=1cm] (8, 8);
	\end{tikzpicture}
\end{figure}
\end{enumerate}

\end{document}